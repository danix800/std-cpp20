% 31.7.1 General [atomics.ref.generic.general]
\paragraph{}
<++>

\paragraph{}
<++>

\paragraph{}
<++>

\paragraph{}
<++>

