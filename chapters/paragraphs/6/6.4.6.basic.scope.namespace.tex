% 6.4.6 Namespace scope [basic.scope.namespace]
\paragraph{} % 1
一个\nt{namespace-definition}的声明区为其\nt{namespace-body}。声明于一个
\nt{namespace-body}中的实体称为命名空间的\df{成员},且由这些声明引入到命名空间的
声明区中的名字称为命名空间的\df{成员名}。一个命名空间成员名具有命名空间作用域。
其潜在作用域从名字的声明点(\ref{basic.scope.pdecl})开始包括其命名空间;且对每
一个指定成员命名空间的\nt{using-directive}(\ref{namespace.udir}),该成员的潜在
作用域包括跟在成员声明点之后的\nt{using-directive}的潜在作用域。

\begin{example}
  \begin{lstlisting}
    namespace N {
      int i;
      int g(int a) { return a; }
      int j();
      void q();
    }
    namespace { int l=1; }
    // the potential scope of l is from its point of declaration to the end of the translation unit

    namespace N {
      int g(char a) { // overloads N::g(int)
        return l+a; // l is from unnamed namespace
      }

      int i; // error: duplicate definition
      int j(); // OK: duplicate function declaration
      int j() { // OK: definition of N::j()
        return g(i);
      }
      int q(); // error: different return type
    }
  \end{lstlisting}
\end{example}

\paragraph{} % 2
如果一个翻译单元\nt{Q}导入到一个翻译单元\nt{R}(\ref{module.import})中,一个声
明于\nt{Q}中具有命名空间作用域的名字\nt{X}的潜在作用域扩展为包含\nt{R}中跟在
(直接或间接)导入\nt{R}的第一个\nt{module-import-declaration}或
\nt{module-declaration}之后对应命名空间作用域的部分,如果
\begin{enumerate}
  \item \nt{X}不具有内部链接,且
  \item \nt{X}声明于\nt{Q}中\nt{module-declaration}(如有)之后,且
  \item 要么\nt{X}被导出,要么\nt{Q}和\nt{R}处于同一模板中。
\end{enumerate}

\begin{note}
  一个\nt{module-import-declaration}递归地导入所命名翻译单元和其中的
  \nt{module-import-declaration}所导出的任何模块。

  \begin{example}

    翻译单元\#1:
  \begin{lstlisting}
    export module Q;
    export int sq(int i) { return i*i; }
  \end{lstlisting}
    翻译单元\#2:
  \begin{lstlisting}
    export module R;
    export module Q;
  \end{lstlisting}
    翻译单元\#3:
  \begin{lstlisting}
    import R;
    int main() { return sq(9); }      // OK: sq from module Q
  \end{lstlisting}

  \end{example}

\end{note}

\paragraph{} % 3
命名空间成员也可以在\tm{::}作用域解析运算符(\ref{expr.prim.id.qual})应用到其命
名空间的名字或在一个\nt{using-directive}中指定一个命名空间成员的命名空间的名字之
后来引用;见\ref{namespace.qual}。

\paragraph{} % 4
一个翻译单元的最外层声明区也是一个命名空间,称为\df{全局命名空间}。一个声明于全
局命名空间中的名字具有\df{全局命名空间作用域}(也称\df{全局作用域})。这种名字的
潜在作用域始于其声明点(\ref{basic.scope.pdecl})并终止于其声明区的翻译单元结束
处。具有全局命名空间作用域的名字称为\df{全局名}。
