% 6.7.5.5.3 Deallocation functions [basic.stc.dynamic.deallocation]
\paragraph{} % 1
释放函数应该是类成员函数或全局函数;如果翻译函数声明于一个非全局作用域的命名空间
作用域中或在全局作用域中声明为静态的则程序为\illform{}。

\paragraph{} % 2
\begin{sloppypar}
一个释放函数是一个\df{销毁运算符delete},如果它具有至少两个形参且第二个形参类型
为\tm{std::destroying\_delete\_t}。一个销毁运算符delete应该是一个类成员函数,名
为\tm{operator delete}。
\end{sloppypar}

\begin{note}
  数组删除不能使用一个销毁运算符delete。
\end{note}

\paragraph{} % 3
每一个释放函数应该返回\tm{void}。如果函数是一个声明于类\tm{C}中的销毁运算符
delete,则其第一个形参类型应该是\tm{C*};否则其第一个形参类型应该是\tm{void*}。
一个释放函数可能具有多于一个形参。一个\df{常规释放函数}是一种释放函数,其第一个
形参之后的形参为
\begin{enumerate}
  \item 一个\tm{std::destroying\_delete\_t}类型的可选形参,然后是
  \item 一个\tm{std::size\_t}\footnote{全局\tm{operator delete(void*,
        std::size\_t)}排除了一个分配函数\tm{void operator new (std::size\_t,
        std::size\_t)}用作定点分配函数(\ref{diff.cpp11.basic})。}类型的可选形
        参,然后是
  \item 一个\tm{std::align\_val\_t}类型的可选形参。
\end{enumerate}
一个销毁运算符delete应该是一个常规释放函数。一个释放函数可能是一个函数模板的实
例。第一个形参和返回类型均不应该依赖于一个模板形参。一个释放函数模板应该具有两个
或多个函数形参。一个模板实例无论其签名如何都不会是一个常规释放函数。

\paragraph{} % 4
如果一个释放函数由抛出一个异常而终止,则行为未定义。提供给一个释放函数的第一个实
参的值可以是一个零指针值;如果是,且如果释放函数为标准库所提供,则调用无效果。

\paragraph{} % 5
如果传递给一个标准库中释放函数的实参是一个非零指针值(\ref{basic.compound}),则
释放函数应该释放该指针所引用的存储,结束存储区的期限。
