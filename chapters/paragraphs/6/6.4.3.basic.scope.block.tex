% 6.4.3 Block scope [basic.scope.block]
\paragraph{} % 1
声明于一个块中的名字局部于该块;具有\df{块作用域}。其潜在作用域始于其声明点
(\ref{basic.scope.pdecl})并终止于其块结束处。声明于块作用域的变量为\df{局部变
量}。

\paragraph{} % 2
声明于一个\nt{exception-declaration}的名字局部于该\nt{handler},且不能在
\nt{handler}的最外层块中重声明。

\paragraph{} % 3
声明于\nt{init-statement},\nt{for-range-declaration}和\tm{if},\tm{while},
\tm{for}和\tm{switch}语句条件中的名字局部于该\tm{if},\tm{while},\tm{for}和
\tm{switch}语句(包括受控语句),且不能在该语句的后续条件中重声明,也不能在受控
语句的最外层块(或对于\tm{if}语句的任何最外层块)中重声明。

\begin{example}
  \begin{lstlisting}
    if (int x = f()) {
      int x;              // error: redeclaration of x
    }
    else {
      int x;              // error: redeclaration of x
    }
  \end{lstlisting}
\end{example}
