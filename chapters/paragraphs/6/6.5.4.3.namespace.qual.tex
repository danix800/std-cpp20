% 6.5.4.3 Namespace members [namespace.qual]
\paragraph{} % 1
如果一个\nt{qualified-id}的\nt{nested-name-specifier}指定一个命名空间(包括
\nt{nested-name-specifier}是\tm{::}的情形,即指定全局命名空间),则
\nt{nested-name-specifier}之后所指定的名字在该命名空间作用域中查询。一个
\nt{template-id}的\nt{template-argument}中的名字在整个\nt{postfix-expression}所
出现的上下文中查询。

\paragraph{} % 2
对一个命名空间\tm{X}和名字\tm{m},命名空间限定查询集\nt{S(X,m)}定义如下:设
\nt{S\textprime{}(X,m)}为\tm{m}在\tm{X}和\tm{X}的内联命名空间集合
(\ref{namespace.def})中所有声明的集合,\tm{X}的潜在作用域
(\ref{basic.scope.namespace})将包含这样的命名空间,在其中\tm{m}在
\nt{nested-name-specifier}的位置声明。如果\nt{S\textprime{}(X,m)}非空,则
\nt{S(X,m)}为\nt{S\textprime{}(X,m)};否则,\nt{S(X,m)}为\nt{S(N\tsub{i},m)}的并
集,\nt{N\tsub{i}}为\tm{X}和其内联命名空间集合中使用\nt{using-directive}指定的所
有命名空间。

\paragraph{} % 3
给定\tm{X::m}(其中\tm{X}为用户声明命名空间),或给定\tm{::m}(其中\tm{X}为全局
命名空间),如果\nt{S(X,m)}为空集,则程序为\illform{}。否则,如果\nt{S(X,m)}恰有
一个成员,或者如果引用的上下文为一个\nt{using-declaration}
(\ref{namespace.udecl}),则\nt{S(X,m)}为\tm{m}声明的所需集合。否则如果\tm{m}的
使用不是允许从\nt{S(X,m)}中选择一个唯一声明的那个,则程序为\illform{}。

\begin{example}
  \begin{lstlisting}
    int x;

    namespace Y {
      void f(float);
      void h(int);
    }

    namespace Z {
      void h(double);
    }

    namespace A {
      using namespace Y;
      void f(int);
      void g(int);
      int i;
    }

    namespace B {
      using namespace Z;
      void f(char);
      int i;
    }

    namespace AB {
      using namespace A;
      using namespace B;
      void g();
    }

    void h()
    {
      AB::g();      // g is declared directly in AB, therefore S is {AB::g()} and AB::g() is chosen

      AB::f(1);     // f is not declared directly in AB so the rules are applied recursively to A and B;
                    // namespace Y is not searched and Y::f(float) is not considered;
                    // S is {A::f(int), B::f(char)} and overload resolution chooses A::f(int)

      AB::f('c');   // as above but resolution chooses B::f(char)

      AB::x++;      // x is not declared directly in AB, and is not declared in A or B, so the rules
                    // are applied recursively to Y and Z, S is {} so the program is ill-formed

      AB::i++;      // i is not declared directly in AB so the rules are applied recursively to A and B,
                    // S is {A::i, B::i} so the use is ambiguous and the program is ill-formed

      AB::h(16.8);  // h is not declared directly in AB and not declared directly in A or B so the rules
                    // are applied recursively to Y and Z, S is {Y::h(int), Z::h(double)} and
                    // overload resolution chooses Z::h(double)
    }
  \end{lstlisting}
\end{example}

\paragraph{} % 4
\begin{note}
  同一个声明被找到多次不是一种歧义(因为其仍是一个唯一声明)。

  \begin{example}
    \begin{lstlisting}
    namespace A {
      int a;
    }

    namespace B {
      using namespace A;
    }

    namespace C {
      using namespace A;
    }

    namespace BC {
      using namespace B;
      using namespace C;
    }

    void f()
    {
      BC::a++;            // OK: S is {A::a, A::a}
    }

    namespace D {
      using A::a;
    }

    namespace BD {
      using namespace B;
      using namespace D;
    }

    void g()
    {
      BD::a++;            // OK: S is {A::a, A::a}
    }
    \end{lstlisting}
  \end{example}

\end{note}

\paragraph{} % 5
\begin{example}
  因为每一个被引用的命名空间最多被搜索一次,以下示例是\wellform{}的:
  \begin{lstlisting}
    namespace B {
      int b;
    }

    namespace A {
      using namespace B;
      int a;
    }

    namespace B {
      using namespace A;
    }

    void f()
    {
      A::a++;             // OK: a declared directly in A, S is {A::a}
      B::a++;             // OK: both A and B searched (once), S is {A::a}
      A::b++;             // OK: both A and B searched (once), S is {B::b}
      B::b++;             // OK: b declared directly in B, S is {B::b}
    }
  \end{lstlisting}
\end{example}

\paragraph{} % 6
在限定命名空间成员名查询过程中,如果查询找到成员的多于一个声明,且如果一个声明引
入一个类名或枚举名而其他声明引入要么相同的变量,相同的\enumr{},要么是一组函数,
则非类型名隐藏类或枚举名当且仅当声明来自同一命名空间中;否则(声明来自不同命名空
间中),程序为\illform{}。

\begin{example}
  \begin{lstlisting}
    namespace A {
      struct x { };
      int x;
      int y;
    }

    namespace B {
      struct y { };
    }

    namespace C {
      using namespace A;
      using namespace B;
      int i = C::x;     // OK, A::x (of type int)
      int j = C::y;     // ambiguous, A::y or B::y
    }
  \end{lstlisting}
\end{example}

\paragraph{} % 7
在一个命名空间成员的声明中,其中\nt{declarator-id}是一个\nt{qualified-id},给定
命名空间成员的\nt{qualified-id}形如\par
\qquad\nt{nested-name-specifier unqualified-id}\par
则\nt{unqualified-id}应该确定由\nt{nested-name-specifier}所表示的命名空间或该命
名空间的内联命名空间集合(\ref{namespace.def})元素的一个成员。

\begin{example}
  \begin{lstlisting}
    namespace A {
      namespace B {
        void f1(int);
      }
      using namespace B;
    }
    void A::f1(int){ }  // error: f1 is not a member of A
  \end{lstlisting}
\end{example}

然而,在这样的命名空间成员声明中,\nt{nested-name-specifier}可能依赖于
\nt{using-directive}来隐式提供该\nt{nested-name-specifier}的初始部分。

\begin{example}
  \begin{lstlisting}
    namespace A {
      namespace B {
        void f1(int);
      }
    }

    namespace C {
      namespace D {
        void f1(int);
      }
    }

    using namespace A;
    using namespace C::D;
    void B::f1(int){ }  // OK, defines A::B::f1(int)
  \end{lstlisting}
\end{example}
