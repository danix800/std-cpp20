% 6.7.5.1 General [basic.stc.general]
\paragraph{} % 1
\df{存储期}指一个对象的定义包含对象存储的最小潜在生命期的属性。存储期由用于创建
对象的结构所确定,为以下之一:
\begin{enumerate}
  \item 静态存储期
  \item 线程存储期
  \item 自动存储期
  \item 动态存储期
\end{enumerate}

\paragraph{} % 2
静态,线程和自动存储期关联于声明(\ref{basic.def})所引入对象和实现隐式创建
(\ref{class.temporary})的对象。动态存储期关联于一个\nt{new-expression}
(\ref{expr.new})所创建对象。

\paragraph{} % 3
存储期类别对引用也适用。

\paragraph{} % 4
当存储区期限终止时,表示存储区任何部分地址的所有指针值成为无效指针值
(\ref{basic.compound})。通过无效指针值进行解引用或者传递无效指针值到释放函数具
有未定义行为。无效指针值的任何其他使用具有实现定义行为。\footnote{实现可以定义拷
贝无效指针值引起一个系统生成的运行时错。}
