% 6.7.5.5.1 General [basic.stc.dynamic.general]
\paragraph{} % 1
对象可在程序执行期间(\ref{intro.execution})使用\nt{new-expression}
(\ref{expr.new})动态地创建,使用\nt{delete-expression}(\ref{expr.delete})动
态地销毁。一个\cpp{}实现通过全局\df{分配函数}\tm{operator new}和\tm{operator
new[]}以及全局\df{释放函数}\tm{operator delete}和\tm{operator delete[]}提供动态
存储的访问和管理。

\begin{note}
  \ref{new.delete.placement}中所述的非分配形式不进行分配和释放。
\end{note}

\paragraph{} % 2
标准库提供了全局分配和释放函数的缺省定义。某些全局分配和释放函数是可替换的
(\ref{new.delete})。一个\cpp{}程序应该提供至多一个可替换分配或释放函数的定义。
任何这样的函数定义替换标准库所提供的缺省版本(\ref{replacement.functions})。以
下分配和释放函数(\ref{support.dynamic})在一个程序的每一个翻译单元的全局作用域
中隐式声明。
\begin{codeenv}
  \code{[[nodiscard]] void* operator new(std::size\_t);}
  \code{[[nodiscard]] void* operator new(std::size\_t, std::align\_val\_t);}
  \code{}
  \code{void operator delete(void*) noexcept;}
  \code{void operator delete(void*, std::size\_t) noexcept;}
  \code{void operator delete(void*, std::align\_val\_t) noexcept;}
  \code{void operator delete(void*, std::size\_t, std::align\_val\_t) noexcept;}
  \code{}
  \code{[[nodiscard]] void* operator new[](std::size\_t);}
  \code{[[nodiscard]] void* operator new[](std::size\_t, std::align\_val\_t);}
  \code{}
  \code{void operator delete[](void*) noexcept;}
  \code{void operator delete[](void*, std::size\_t) noexcept;}
  \code{void operator delete[](void*, std::align\_val\_t) noexcept;}
  \code{void operator delete[](void*, std::size\_t, std::align\_val\_t)
        noexcept;}
\end{codeenv}
这些隐式声明只引入函数名\tm{operator new},\tm{operator new[]},\tm{operator
delete}和\tm{operator delete[]}。

\begin{note}
  这些隐式声明不引入名字\tm{std},\tm{std::size\_t},\tm{std::align\_val\_t}或任
  何其他标准库用于声明这些名字的名字。因此,一个\nt{new-expression},
  \nt{delete-expression}或引用这些函数之一的函数调用而未导入或包含头\tm{<new>}
  (\ref{new.syn})也是有定义的。然而,引用\tm{std}或\tm{std::size\_t}或
  \tm{std::align\_val\_t}是\illform{}的,除非名字通过导入或包含适当的头来声明。
\end{note}

分配和/或释放函数也可以为任何类声明和定义(\ref{class.free})。

\paragraph{} % 3
如果一个分配或释放函数的行为不满足\ref{basic.stc.dynamic.allocation}和
\ref{basic.stc.dynamic.deallocation}中所指定的语义约束则行为未定义。
