% 6.8.3 Compound types [basic.compound]
\paragraph{} % 1
复合类型可按以下方式构建:
\begin{enumerate}
  \item 一个给定类型对象的\df{数组},\ref{dcl.array};
  \item \df{函数},具有一个给定类型的形参,返回\tm{void}或者一个给定类型的引用或
        对象,\ref{dcl.fct};
  \item 指向\nt{cv} \tm{void}或一个给定类型的对象或函数(含类的静态成员)的
        \df{指针},\ref{dcl.ptr}
  \item 一个给定类型对象或函数的\df{引用},\ref{dcl.ref}。存在两种引用类型:
        \begin{enumerate}
          \item \lvalue{}引用
          \item \rvalue{}引用
        \end{enumerate}
  \item \df{类},包含一个多种类型对象的序列(第\ref{class}条),一组类型,枚举和
        函数用于操作这些对象(\ref{class.mfct}),以及一组访问这些实体的约束
        (\ref{class.access});
  \item \df{联合},一种能够在不同时间包含不同类型对象的类,\ref{class.union};
  \item \df{枚举},由一组命名常量值组成。每一个不同枚举构成一个不同的\nt{枚举类
        型},\ref{dcl.enum};
  \item \df{非静态类成员指针},\footnote{静态类成员为对象或函数,指向它们的指针
        为指向对象和函数的普通指针。}标识一个给定类对象中一个给定类型的成员,
        \ref{dcl.mptr}。数据成员指针和成员函数指针统称为\df{成员指针}类型。
\end{enumerate}

\paragraph{} % 2
这些构建类型的方法可以递归的应用;约束在\ref{dcl.meaning}中提及。构建一个类型使
得其对象表示的字节数超过类型\tm{std::size\_t}(\ref{support.types})能表示的最大
值是\illform{}的。

\paragraph{} % 3
指向\nt{cv} \tm{void}或指向对象类型的指针的类型称为\df{对象指针类型}。

\begin{note}
  但指向\tm{void}的指针不具有对象指针类型,因为\tm{void}不是一个对象类型。
\end{note}

一个表示函数的指针的类型称为\df{函数指针类型}。一个指向\tm{T}类型对象的指针称为
“指向\tm{T}的指针”。

\begin{example}
  一个\tm{int}类型对象的指针称为“指向\tm{int}的指针”,一个类\tm{X}的对象的指针称
  为“指向\tm{X}的指针”。
\end{example}

除了静态成员的指针,称为“指针”的文字对成员指针并不适用。指向不完整类型的指针是允
许的,虽然存在其能做什么的限制(\ref{basic.align})。指针类型的每一个值为以下之
一:
\begin{enumerate}
  \item 一个指向对象或函数的\df{指针}(指针称为\df{指向}对象或函数),或者
  \item \df{跨过}一个对象\df{结尾的指针}(\ref{expr.add}),或者
  \item 该类型的\df{零指针值},或者
  \item 一个\df{无效指针值}。
\end{enumerate}
一个指向对象或者跨过对象结尾的指针类型的值\df{表示一个地址},分别为对象所占内存
(\ref{intro.memory})首字节的地址\footnote{对一个不在其生命期内的对象,这指的是
其将会占据或者曾经占据的内存首字节。}或者对象所占存储尾部之后的内存的第一个字节
的地址。

\begin{note}
  一个跨过对象尾部的指针(\ref{expr.add})不认为是指向该对象类型的不相关对象,即
  使某个不相关对象位于该地址。当其表示的存储到达其存储期终束时,一个指针值变得无
  效;见\ref{basic.stc}。
\end{note}

为了指针算术(\ref{expr.add})和比较(\ref{expr.rel},\ref{expr.eq})的目的,一
个跨过\nt{n}个元素数组的最后一个元素尾部的指针认为与一个指向假想的第\nt{n}个数组
元素等价,而一个非数组元素的\tm{T}类型对象认为属于一个\tm{T}类型一个元素的数组。
指针类型的值表示由实现定义。布局兼容类型的指针应该具有相同的值表示和对齐要求
(\ref{basic.align})。

\begin{note}
  过度对齐类型的指针(\ref{basic.align})无特殊表示,但其有效值范围由扩展对齐要
  求所限制。
\end{note}

\paragraph{} % 4
两个对象\nt{a}和\nt{b}是\df{指针可互转的},如果:
\begin{enumerate}
  \item 它们是相同对象,或者
  \item 其中一个是联合类型,另一个是该对象的非静态数据成员(\ref{class.union})
        ,或者
  \item 其中一个是布局兼容类对象,另一个是该对象的第一个非静态数据成员,或者如果
        对象不具有非静态数据成员,该对象的任何基类子对象(\ref{class.mem}),或
        者
  \item 存在对象\nt{c}使得\nt{a}和\nt{c}为指针可互转的,且\nt{c}和\nt{b}为指针可
        互转的。
\end{enumerate}
如果两个对象为指针可互转的,则它们具有相同的地址,且可以通过
\tm{reinterpret\_cast}(\ref{expr.reinterpret.cast})从一个指针获取另一个的指
针。

\begin{note}
  一个数组对象和其第一个元素不是指针可互转的,即使其具有相同地址。
\end{note}

\paragraph{} % 5
一个\nt{cv} \tm{void}的指针可用于指向未知类型的对象。这样的指针应该能够存储任意
对象指针。一个\nt{cv} \tm{void*}类型的对象应该与\nt{cv} \tm{char*}具有相同的表示
和对齐要求。
