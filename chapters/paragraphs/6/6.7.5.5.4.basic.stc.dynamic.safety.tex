% 6.7.5.5.4 Safely-derived pointers [basic.stc.dynamic.safety]
\paragraph{} % 1
一个\df{可追踪指针对象}指
\begin{enumerate}
  \item 一个对象指针类型(\ref{basic.compound})的对象,或者
  \item 一个整型对象,至少和\tm{std::intptr\_t}一样大,或者
  \item 一个窄字符类型(\ref{basic.fundamental})数组的元素序列,其中序列的大小
        和对齐匹配某些对象指针类型的大小和对齐。
\end{enumerate}

\paragraph{} % 2
一个指针值是一个动态存储期对象的\df{安全派生指针},仅当指针值具有对象指针类型且
是以下之一:
\begin{enumerate}
  \item 调用\cpp{}标准库实现的\tm{::operator new(std::size\_t)}或者
        \tm{::operator new(std::}\linebreak \tm{size\_t, std::align\_val\_t)}的
        返回值;\footnote{该子条款不对通过指向不由\tm{::operator new}分配内存的
        指针进行解引用施加限制。这维护了许多\cpp{}实现使用其他语言所写的二进制库
        和组件的能力。特别的,这适用于\c{}二进制程序,因为通过指向由
        \tm{std::malloc}分配内存的指针进行解引用不受限。}
  \item 取通过一个安全派生指针值解引用所产生的\lvalue{}所表示对象(或其子对象之
        一)的地址的结果;
  \item 使用一个安全派生指针值的有定义指针算术(\ref{expr.add})的结果;
  \item 一个安全派生指针值的有定义指针转换(\ref{conv.ptr},
        \ref{expr.type.conv},\ref{expr.static.cast},\ref{expr.cast})的结果;
  \item 一个安全派生指针值的\tm{reinterpret\_cast}的结果;
  \item 一个安全派生指针值的整数表示的\tm{reinterpret\_cast}的结果;
  \item 值拷贝自一个可追踪指针对象的对象值,在拷贝时源对象包含一个安全派生指针值
        的拷贝。
\end{enumerate}

\paragraph{} % 3
一个整数值是一个\df{安全派生指针的整数表示},仅当其类型至少和\tm{std::intptr\_t}
一样大且是以下之一:
\begin{enumerate}
  \item 一个安全派生指针值的\tm{reinterpret\_cast}的结果;
  \item 一个安全派生指针值的整数表示的有效转换结果;
  \item 值拷贝自一个可追踪指针对象的对象值,在拷贝时源对象包含一个安全派生指针值
        的整数表示;
  \item 一个加法或按位操作的结果,其操作数之一是一个安全派生指针\tm{P}的整数表
        示,如果该结果被\tm{reinterpret\_cast}转换与可从
        \tm{reinterpret\_cast<void*>(P)}计算而得的一个安全派生指针相等。
\end{enumerate}

\paragraph{} % 4
一个实现可以具有\df{放松的指针安全性},此时一个指针值的有效性不依赖于其是否是一
个安全派生指针值。可选的,实现可以具有\df{严格的指针安全性},此时一个引用动态存
储期对象的非安全派生指针值是一个无效指针值,除非所引用完整对象之前已经声明为可达
(\ref{util.dynamic.safety})。

\begin{note}
  使用一个无效指针值的效果(包括传递给一个释放函数)是未定义的,见
  \ref{basic.stc}。即使不安全派生指针值与某些安全派生指针值相等也是。
\end{note}

一个实现具有放松还是严格指针安全性由实现来定义。
