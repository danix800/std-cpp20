% 6.9.3.1 main function [basic.start.main]
\paragraph{} % 1
一个程序应该包含一个称为\tm{main}的全局函数附加于全局模块。执行程序将启动主执行
线程(\ref{intro.multithread},\ref{thread.threads}),在该线程中调用\tm{main}函
数。自由式环境中的程序是否需要定义\tm{main}函数由实现定义。

\begin{note}
  在自由式环境中,启动和终止由实现来定义;启动包含具有静态存储期命名空间作用域对
  象的构造函数的执行;终止包含具有静态存储期对象的析构函数的执行。
\end{note}

\paragraph{} % 2
实现不应预定义\tm{main}函数。该函数不得重载。其类型应具有\cpp{}语言链接,并且应
具有一个声明为\tm{int}类型的返回类型,否则其类型由实现来定义。一个实现应允许
\begin{enumerate}
  \item 一个返回\tm{int}的函数\tm{()},以及
  \item 一个返回\tm{int}的函数\tm{(int, char}的指针的指针\tm{)}
\end{enumerate}
作为\tm{main}的类型(\ref{dcl.fct})。在后一种形式中,为说明的目的,第一个函数形
参称为\tm{argc},第二个函数形参称为\tm{argv},其中\tm{argc}为程序运行所在环境传
递给程序的实参个数。如果\tm{argc}为非零,则这些参数应该在\tm{argv[0]}到
\tm{argv[argc-1]}中以指向零结尾多字节字符串(NTMBS)(\ref{multibyte.strings})
首字符的指针来提供,且\tm{argv[0]}应该是指向表示用来调用程序名字的NTMBS或\tm{""}
的首字符指针。\tm{argc}的值应该为非负值。\tm{argv[argc]}的值应该0。

\begin{note}
  建议在\tm{argv}之后添加任何其他(可选)参数。
\end{note}

\paragraph{} % 3
函数\tm{main}不应在程序内使用。 \tm{main}的链接(\ref{basic.link})实现来定义。
将\tm{main}定义为删除或将\tm{main}声明为\tm{inline},\tm{static}或\tm{constexpr}
的程序为\illform{}。函数\tm{main}不能是协程(\ref{dcl.fct.def.coroutine})。不应
使用\nt{linkage-specification}(\ref{dcl.link})声明\tm{main}函数。在全局作用域
内声明变量\tm{main}的程序,或在全局作用域内声明附加到命名模块的函数\tm{main}的程
序,或使用\c{}语言链接(在任何命名空间中)声明名称\tm{main}的程序都是\illform{}
。除此之外名称\tm{main}不保留。

\begin{example}
  成员函数,类和枚举可以称为\tm{main},其他命名空间中的实体也可以。
\end{example}

\paragraph{} % 4
在不离开当前块的情况下终止程序(例如,通过调用函数\tm{std::exit(int)}
(\ref{support.start.term}))不会销毁具有自动存储期(\ref{class.dtor})的任何对
象。如果在销毁具有静态或线程存储期对象的过程中调用\tm{std::exit}结束程序,则该程
序具有未定义行为。

\paragraph{} % 5
\tm{main}中的\tm{return}语句(\ref{stmt.return})具有退出\tm{main}函数(销毁所有
自动存储期对象)并使用返回值作为实参调用\tm{std::exit}函数的效果。如果控制从
\tm{main}的\nt{compound-statement}末尾流出,则效果等效于具有操作数为0的
\tm{return}(另请参见\ref{except.handle})。
