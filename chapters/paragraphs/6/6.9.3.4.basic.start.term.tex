% 6.9.3.4 Termination [basic.start.term]
\paragraph{} % 1
所构建的具有静态存储期对象(\ref{dcl.init})被销毁,且在\tm{std::atexit}中注册的
函数作为\tm{std::exit}(\ref{support.start.term})函数调用的一部分进行调用。对
\tm{std::exit}的调用后序于销毁和所注册函数。

\begin{note}
  从\tm{main}返回调用\tm{std::exit}(\ref{basic.start.main})。
\end{note}

\paragraph{} % 2
给定线程中所构造具有线程存储期对象因从该线程初始函数返回和该线程中调用
\tm{std::exit}而销毁。销毁该线程内具有线程期的所有构造对象强先发生于在销毁具有静
态存储期的任何对象。

\paragraph{} % 3
如果构造函数的完成或具有静态存储期对象的动态初始化强先发生于其他对象,则启动第一
个析构函数前序于启动第二个析构函数。如果具有线程存储期对象的构造函数的完成或动态
初始化的执行前序于另一个,则第二个析构函数的完成前序于第一个析构函数的初始化。如
果对象是静态初始化的,则该对象的销毁顺序与该对象被动态初始化的顺序相同。对于数组
或类类型对象,该对象的所有子对象都必须先销毁,然后再销毁在构建子对象期间初始化的
静态存储期的任何块作用域对象。如果通过异常退出具有静态或线程存储期对象的销毁,则
调用函数\tm{std::terminate}(\ref{except.terminate})。

\paragraph{} % 4
如果函数包含已销毁静态或线程存储期块作用域对象,并且在销毁具有静态或线程存储期对
象过程中调用了该函数,如果控制流通过先前销毁的块作用域对象的定义,则程序具有未定
义行为。同样,如果在销毁对象之后间接(即通过指针)使用该块作用域对象,则该行为也
是未定义的。

\paragraph{} % 5
如果具有静态存储期对象初始化的完成强先发生于对\tm{std::atexit}的调用(见
\tm{<cstdlib>},\ref{support.start.term}),则传递给\tm{std::atexit}的函数的调用
前序于对象的析构函数调用。如果对具有静态存储期对象的初始化强先发生于对
\tm{std::atexit}的调用,则传递给\tm{std::atexit}的函数的调用前序于对对象的析构函
数的调用。如果对\tm{std::atexit}的调用强先发生于另一个对\tm{std::atexit}的调用,
则传递给第二个\tm{std::atexit}的函数的调用将前序于传递给第一个\tm{std::atexit}的
函数的调用。

\paragraph{} % 6
如果在信号处理程序(\ref{support.runtime})中使用了不允许使用的标准库对象或者函
数,而没有先发生于(\ref{intro.multithread})具有静态存储期对象的销毁并执行
\tm{std::atexit}所注册函数(\ref{support.start.term})的完成,则该程序具有未定义
行为。

\begin{note}
  如果存在没有先发生于销毁对象的静态存储期对象的使用,则该程序具有未定义的行为。
  在调用\tm{std::exit}或从\tm{main}退出之前终止每个线程就可以满足这些要求,但不
  是必需条件。这些要求允许线程管理器作为静态存储期对象。
\end{note}

\paragraph{} % 7
调用在\tm{<cstdlib>}(\ref{cstdlib.syn})中声明的\tm{std::abort()}函数可以终止程
序,而无需执行任何析构函数,也无需调用传递给\tm{std::atexit()}或
\tm{std::at\_quick\_exit()}的函数。
