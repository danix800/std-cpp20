% 6.5.1 General [basic.lookup.general]
\paragraph{} % 1
名字查询规则一律适用于所有名字(包括\nt{typedef-name}(\ref{dcl.typedef}),
\nt{namespace-name}(\ref{basic.namespace})和\nt{class-name}
(\ref{class.name})),只要在一条特定规则所讨论的上下文中语法允许这样的名字。名
字查询将一个名字的使用与该名字的一组声明(\ref{basic.def})关联起来。如果名字查
询所找到的声明全部表示函数或函数模板,则这些声明称作形成一个\df{重载集合}。名字
查询所找到的声明应该要么全部表示同一个实体,要么形成一个重载集合。重载解析
(\ref{over.match},\ref{over.over})发生在名字查询成功之后。访问规则
(\ref{class.access})只在名字查询和函数重载解析(如适用)成功之后考虑一次。只有
在名字查询,函数重载解析(如适用)和访问检查成功之后该名字的声明及其可达
(\ref{module.reach})重声明所引入的语义属性才用于进一步的表达式处理(第
\ref{expr}条)。

\paragraph{} % 2
一个“在表达式上下文中查询”的名字在找到表达式的作用域中查询。

\paragraph{} % 3
一个类的注入类名(\ref{class.pre})为名字隐藏和查询的目的也被当作是该类的一个成
员。

\paragraph{} % 4
\begin{note}
  \ref{basic.link}中讨论了链接问题。作用域,声明点和名字隐藏的概念在
  \ref{basic.scope}中讨论。
\end{note}
