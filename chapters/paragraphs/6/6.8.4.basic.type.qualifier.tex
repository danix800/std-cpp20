% 6.8.4 CV-qualifiers [basic.type.qualifier]
\paragraph{} % 1
一个在\ref{basic.fundamental}和\ref{basic.compound}中所提及的类型是一个\df{无cv
限定类型}。每一个无cv限定完整或不完整对象类型或者\tm{void}(\ref{basic.types})
类型具有其类型的三个对应cv限定版本:一个\df{const限定}版本,一个\df{volatile限
定}版本以及一个\df{const-volatile限定}版本。一个对象的类型(\ref{intro.object})
包括\nt{decl-specifier-seq}(\ref{dcl.spec})中指定的\nt{cv-qualifier},
\nt{declarator}(\ref{dcl.decl}),\nt{type-id}(\ref{dcl.name})或者当创建对象
时的\nt{new-type-id}(\ref{expr.new})。
\begin{enumerate}
  \item 一个\df{常对象}是一个类型为\tm{const T}的对象或者常对象的不可变子对象。
  \item 一个\df{易失对象}是一个类型为\tm{volatile T}的对象或者易失对象的子对象。
  \item 一个\df{常易失对象}是一个类型为\tm{const volatile T}的对象,常易失对象的
        不可变子对象,一个易失对象的常子对象或者一个常对象的不可变易失子对象。
\end{enumerate}
一个类型的cv限定和无cv限定版本是不同的类型;但它们应该具有相同的表示和对齐要求
(\ref{basic.align})。\footnote{相同表示和对齐要求意味着作为函数实参,函数返回
值和联合的非静态数据成员的可互换性。}

\paragraph{} % 2
除数组类型外,一个复合类型(\ref{basic.compound})不被复合于类型的cv限定符(如果
有)所限定。

\paragraph{} % 3
任何元素为cv限定的数组类型被认为与其元素具有相同的cv限定。

\begin{note}
  应用于一个数组类型的cv限定符附属于其底层类型,因此记号“\nt{cv} \tm{T}”,其中
  \tm{T}为一个数组类型,指的是其元素如此限定的数组(\ref{dcl.array})。
\end{note}

\begin{example}
  \begin{lstlisting}
    typedef char CA[5];
    typedef const char CC;
    CC arr1[5] = { 0 };
    const CA arr2 = { 0 };
  \end{lstlisting}
  \tm{arr1}和\tm{arr2}的类型均为“5个\tm{const char}的数组”,并且数组类型是常限定
  的。
\end{example}

\paragraph{} % 4
\begin{note}
  见\ref{dcl.fct}和\ref{class.this}关于具有\nt{cv-qualifier}的函数类型。
\end{note}

\paragraph{} % 5
在cv限定符上存在一个偏序,使得一个类型可以称为比另一个类型“\df{更多限定}”。表
\ref{tab:basic.type.qualifier.rel}列出构成此偏序的关系。

\begin{table}[!ht]
  \centering
  \caption{\tm{const}和\tm{volatile}上的关系[tab:basic.type.qualifier.rel]}
  \begin{tabular}{|ccc|}
    \hline
    \df{无cv限定符} & < & \tm{const}                                          \\
    \df{无cv限定符} & < & \tm{volatile}                                       \\
    \df{无cv限定符} & < & \tm{const volatile}                                 \\
    \tm{const}      & < & \tm{const volatile}                                 \\
    \tm{volatile}   & < & \tm{const volatile}                                 \\
    \hline
  \end{tabular}
  \label{tab:basic.type.qualifier.rel}
\end{table}

\paragraph{} % 6
在本文件中,用于类型描述中的记号\nt{cv}(或\nt{cv1},\nt{cv2}等等)表示任意cv限
定符集合,即\tm{\{const\}},\tm{\{volatile\}},\tm{\{const, volatile\}}之一或者
空集。对于一个类型\nt{cv} \tm{T},该类型的顶层\nt{cv-qualifier}由\nt{cv}来表示。

\begin{example}
  对应于\nt{type-id} \tm{const int\&}的类型没有顶层cv限定符。对应于\nt{type-id}
  \tm{volatile int * const}的类型具有顶层cv限定符\tm{const}。对一个类类型\tm{C}
  对应于\tm{type-id} \tm{void (C::* volatile)(int) const}的类型具有顶层cv限定符
  \tm{volatile}。
\end{example}
