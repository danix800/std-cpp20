% 6.2 Declarations and definitions [basic.def]
\paragraph{} % 1
一个声明(第\ref{dcl.dcl}条)可以向翻译单元内引入一个或多个名字,或重新声明之前
的声明所引入的名字。如果是则声明指定这些名字的解释和语义属性。一个声明也可以具有
以下效果:
\begin{enumerate}
  \item 一条静态断言(\ref{dcl.pre}),
  \item 控制模板实例化(\ref{temp.explicit}),
  \item 指导构造函数模板实参推导(\ref{temp.deduct.guide}),
  \item 属性使用(\ref{dcl.attr}),以及
  \item 无效果(在\nt{empty-declaration}的情况下)。
\end{enumerate}

\paragraph{} % 2
一个\nt{declaration}所声明的每一个实体也会由该声明进行\df{定义},除了:
\begin{enumerate}
  \item 声明了函数但未指定函数体(\ref{dcl.fct.def}),
  \item 声明含有\tm{extern}说明符(\ref{dcl.stc})或者一个
        \nt{linkage-specification}(\ref{dcl.link})\footnote{出现在
        \nt{linkage-specification}由花括号包含的\nt{declaration-seq}中不会影响声
        明是否为定义。}并且既没有\nt{initializer}也没有\nt{function-body},
  \item 在类定义(\ref{class.mem},\ref{class.static})中声明了非内联静态数据成
        员,
  \item 在类定义外声明了静态数据成员,该变量在类内使用\tm{constexpr}说明符进行了
        定义(该用法已弃用;见\ref{depr.static.constexpr}),
  \item 由一个\nt{elaborated-type-specifier}(\ref{class.name})引入,
  \item 是一个\nt{opaque-enum-declaration}(\ref{dcl.enum}),
  \item 是一个\nt{template-parameter}(\ref{temp.param}),
  \item 不是\nt{function-definition}的\nt{declarator}的函数声明子中的
        \nt{parameter-declaration}(\ref{dcl.fct}),
  \item 是一个\tm{typedef}声明(\ref{dcl.typedef}),
  \item 是一个\nt{alias-declaration}(\ref{dcl.typedef}),
  \item 是一个\nt{using-declaration}(\ref{namespace.udecl}),
  \item 是一个\nt{deduction-guide}(\ref{temp.deduct.guide}),
  \item 是一个\nt{static\_assert-declaration}(\ref{dcl.pre}),
  \item 是一个\nt{attribute-declaration}(\ref{dcl.pre}),
  \item 是一个\nt{empty-declaration}(\ref{dcl.pre}),
  \item 是一个\nt{using-directive}(\ref{namespace.udir}),
  \item 是一个\nt{using-enum-declaration}(\ref{enum.udecl}),
  \item 是一个\nt{template-declaration}(\ref{temp.pre}),其\tm{template-head}
        后没有跟上一个\nt{concept-definition}或一个定义函数,类,变量或静态数据
        成员的\nt{declaration},
  \item 是一个显式实例化声明(\ref{temp.explicit}),或者
  \item 是一个显式特例化(\ref{temp.expl.spec}),其\nt{declaration}不是定义。
\end{enumerate}
一个声明是其所定义实体的\df{定义}。

\begin{example}
  以下仅有一个不是定义:
  \begin{lstlisting}
    int a;                         // defines a
    extern const int c = 1;        // defines c
    int f(int x) { return x + a; } // defines f and defines x
    struct S { int a; int b; }     // defines S, S::a, and S::b
    struct X {                     // defines X
      int x;                       // defines non-static data member x
      static int y;                // declares static data member y
      X() : x(0) { }               // defines a constructor of X
    };
    int X::y = 1;                  // defines X::y
    enum { up, down };             // defines up and down
    namespace N { int d; }         // defines N and N::d
    namespace N1 = N;              // defines N1
    X anX;                         // defines anX
  \end{lstlisting}
  而以下仅为声明:
  \begin{lstlisting}
    extern int a;                  // declares a
    extern const int c;            // declares c
    int f(int);                    // declares f
    struct S;                      // declares S
    typedef int Int;               // declares Int
    extern X anotherX;             // declares anotherX
    using N::d;                    // declares d
  \end{lstlisting}
\end{example}

\paragraph{} % 3
\begin{note}
  某些情况下,\cpp{}实现隐式定义缺省构造函数(\ref{class.default.ctor}),拷贝构
  造函数,移动构造函数(\ref{class.copy.ctor}),拷贝赋值运算符,移动赋值运算符
  (\ref{class.copy.assign})或析构函数(\ref{class.dtor})。
\end{note}

\begin{example}
  给定
  \begin{lstlisting}
    #include <string>

    struct C {
      std::string s;      // std::string is the standard library class (21.3)
    };

    int main() {
      C a;
      C b = a;
      b = a;
    }
  \end{lstlisting}
  实现将隐式定义函数使得\tm{C}的定义等价于
  \begin{lstlisting}
    struct C {
      std::string s;
      C() : s() { }
      C(const C& x) : s(x.s) { }
      C(C&& x) : s(static_cast<std::string&&>(x.s) { }
         //    : s(std::move(x.s)) { }
      C& operator=(const C& x) { s = x.s; return *this; }
      C& operator=(C&& x) { s = static_cast<std::string&&>(x.s); return *this; }
          //              { s = std::move(x.s); return *this; }
      ~C() { }
    };
  \end{lstlisting}
\end{example}

\paragraph{} % 4
\begin{note}
  类名也可能由一个\nt{elaborated-type-specifier}(\ref{dcl.type.elab})隐式声
  明。
\end{note}

\paragraph{} % 5
在一个对象的声明中,该对象的类型不应该是一个不完整类型(\ref{basic.types}),一
个抽象类类型(\ref{class.abstract})或者是它们的一个(可能是多维的)数组。
