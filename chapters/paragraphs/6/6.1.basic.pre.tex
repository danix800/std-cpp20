% 6.1 Preamble [basic.pre]
\paragraph{} % 1
\begin{note}
  本条款介绍了\cpp{}语言基本概念。解释了对象和名字间的区别,以及它们与表达式
  \valcat{}的关联。引入声明和定义的概念,并介绍了\cpp{}的类型,作用域,链接和存
  储期概念。讨论了程序启动和终止机制。最后,本条款介绍了语言的基本类型,并列出从
  这些基本类型构造复合类型的方式。
\end{note}

\paragraph{} % 2
\begin{note}
  本条件不覆盖只影响语言单个部分的概念。这些概念在相关条款中讨论。
\end{note}

\paragraph{} % 3
一个\df{实体}指一个值,对象,引用,结构化绑定,函数,\enumr{},类型,类成员,位
域,模板,模板特例化,命名空间或包。

\paragraph{} % 4
一个\df{名字}指对一个表示实体或\df{\lbl}的
\nt{identifier}             (\ref{lex.name}),
\nt{operator-function-id}   (\ref{over.oper}),
\nt{literal-operator-id}    (\ref{over.literal}),
\nt{conversion-function-id} (\ref{class.conv.fct}),或者
\nt{template-id}            (\ref{temp.names})的使用。

\paragraph{} % 5
每一个表示实体的名字由一个\nt{declaration}引入。每一个表示\lbl{}的名字由一条
\tm{goto}语句(\ref{stmt.goto})或一条\tm{labeled-statement}(\ref{stmt.label})
引入。

\paragraph{} % 6
一个\df{变量}由一个除了非静态数据成员的引用声明或一个对象的声明引入。如果有的话
变量的名字表示该引用或对象。

\paragraph{} % 7
一个\df{局部实体}指一个自动存储期变量(\ref{basic.stc.auto}),一个其对应变量为
这样的实体的结构化绑定(\ref{dcl.struct.bind}),或者\tm{*this}对象
(\ref{expr.prim.this})。

\paragraph{} % 8
某些名字表示类型或模板。一般而言,每当遇到一个名字时,在继续分析包含它的程序之前
都需要确定该名字是否表示这些实体之一。确定此信息的过程称为\df{名字查询}
(\ref{basic.lookup})。

\paragraph{} % 9
如果两个名字
\begin{enumerate}
  \item 是由相同字符序列组成的\nt{identifier},或者
  \item 是由相同运算符形成的\nt{operator-function-id},或者
  \item 是由相同类型形成的\nt{conversion-function-id},或者
  \item 是由引用相同类,函数或变量的\nt{template-id}(\ref{temp.type}),或者
  \item 是由相同字面值后缀标识符形成的\nt{literal-operator-id}
        (\ref{over.literal}),
\end{enumerate}
则这两个名字是\df{相同的}。

\paragraph{} % 10
根据每一个翻译单元中对名字所指定的链接(\ref{basic.link}),用于多个翻译单元中的
一个名字可以潜在地引用这些翻译单元中的相同实体。
