% 6.9.2.1 General [intro.multithread.general]
\paragraph{} % 1
一个\df{执行线程}(也称\df{线程})指程序中的单个控制流,包含一个特定的顶层函数的
调用,且递归地包含线程后续的函数调用。

\begin{note}
  当一个线程创建另一个线程时,新线程的顶层函数的初始调用由新线程执行,而不是由调
  用线程执行。
\end{note}

程序中每一个线程可以潜在的访问程序中的每一个对象和函数。\footnote{一个自动或线程
存储期对象(\ref{basic.stc})关联于一个特定线程,只能通过指针或引用
(\ref{basic.compound})被另一个线程间接访问。}在一个宿主式实现中,一个\cpp{}程
序可以具有多于一个线程并发运行。每一个线程的执行按本文件剩余部分所定义进行。整个
程序的执行包括其所有线程的执行。

\begin{note}
  通常执行可以看作是其所有线程的交织进行。然而比如某些类型的原子操作,允许与简单
  的交织不一样,如下文所述。
\end{note}

在一个自由式实现中,程序是否可以具有多个线程由实现定义。

\paragraph{} % 2
对一个不因\tm{std::raise}函数调用所引起的信号处理程序的执行,未指明哪个执行线程
包含该信号处理程序的调用。
