% 6.7.5.5.2 Allocation functions [basic.stc.dynamic.allocation]
\paragraph{} % 1
一个分配函数应该是一个类成员函数或一个全局函数;如果一个分配函数声明于非全局作用
域的命名空间作用域中或者在全局作用域中声明为静态的,则程序为\illform{}。返回类型
应该是\tm{void*}。第一个形参具有\tm{std::size\_t}类型(\ref{support.types})。第
一个形参不应该具有关联缺省实参(\ref{dcl.fct.default})。第一个形参解释为分配所
请求的大小。分配函数可以是一个函数模板。这样的模板应该按上文所指定声明其返回类型
和第一个形参(即模板形参类型不应该用于返回类型和第一个形参类型中)。模板分配函数
应该具有两个或更多的形参。

\paragraph{} % 2
一个分配函数尝试分配所请求大小的存储。如果成功,则返回存储块首地址,存储块长度应
该至少和所请求大小一样大。未指明后续调用分配函数所分配存储的顺序,连续性和初始化
值。即使所请求空间大小为零,请求也可能会失败。如果请求成功,则一个可替换分配函数
的返回值为非零指针值(\ref{basic.compound})\tm{p0},不同于任何之前的返回值
\tm{p1},除非值\tm{p1}曾传递给一个可替换释放函数。更进一步,对
\ref{new.delete.single}和\ref{new.delete.array}中的标准库分配函数,\tm{p0}表示与
调用者可访问的任何其他对象存储不相交的存储块的地址。通过零大小请求返回的指针进行
解引用的效果未定义。\footnote{其目的是使得\tm{operator new()}可以通过调用
\tm{std::malloc()}或者\tm{std::calloc()}来实现,因此规则本质上是一样的。\cpp{}在
要求零请求返回非零指针上与\c{}不同。}

\paragraph{} % 3
对于一个除了保留的定点分配函数(\ref{new.delete.placement})之外的分配函数,一个
成功调用所返回的指针应该表示按以下对齐的存储地址:
\begin{enumerate}
  \item 如果分配函数取一个\tm{std::align\_val\_t}类型的实参,则存储将具有该实参
        值所指定的对齐。
  \item 否则,如果分配函数为\tm{operator new[]},则存储为任何不具有new扩展对齐
        (\ref{basic.align})的对象而对齐且不会比请求大小更大。
  \item 否则,存储对任何不具有new扩展对齐的对象而对齐并具有所请求大小。
\end{enumerate}

\paragraph{} % 4
一个未成功分配存储的分配函数可以调用当前安装的new处理函数(\ref{new.handler}),
如有的话。

\begin{note}
  一个程序提供的分配函数可以使用\tm{std::get\_new\_handler}函数
  (\ref{get.new.handler})来获取当前安装的\tm{new\_handler}的地址。
\end{note}

一个具有不抛出异常规范(\ref{except.spec})的分配函数通过返回一个零指针值来表示
失败。任何其他分配函数永远不会返回零指针值,并只通过抛出匹配\tm{std::bad\_alloc}
(\ref{bad.alloc})类型处理程序的异常(\ref{except.throw})来表示失败。

\paragraph{} % 5
一个全局分配函数只因一个new表达式(\ref{expr.new})来调用,或使用函数调用语法
(\ref{expr.call})来直接调用,或为一个协程状态(\ref{dcl.fct.def.coroutine})分
配存储而间接调用,或通过调用\cpp{}标准库中函数而间接调用。

\begin{note}
  特别的,不会调用一个全局分配函数来为静态存储期(\ref{basic.stc.static})对象,
  线程存储期(\ref{basic.stc.thread})对象或引用,\tm{std::type\_info}类型的对象
  或者一个异常对象分配存储。
\end{note}
