% 6.8.2 Fundamental types [basic.fundamental]
\paragraph{} % 1
存在五种\df{标准有符号整型}:“\tm{signed char}”,“\tm{short int}”,“\tm{int}”,
“\tm{long int}”和“\tm{long long int}”。该列表中的每种类型提供与列表中前一个至少
一样多的存储空间。可能存在实现定义的\nt{扩展有符号整型}。标准和扩展有符号整型统
称\df{有符号整型}。有符号整型的可表示值范围为$-2^{N-1}$到$2^{N-1}-1$(含),其中
\nt{N}称为类型的\df{宽度}。

\begin{note}
  普通的\nt{int}具有执行环境架构推荐的自然大小;其他有符号整型用于特殊需求。
\end{note}

\paragraph{} % 2
对每一个标准有符号整型,存在一个对应(但不同)的\df{标准无符号整型}:
“\tm{unsigned char}”,“\tm{unsigned short int}”,“\tm{unsigned int}”,
“\tm{unsigned long int}”,“\tm{unsigned long long int}”。同样,对每一个扩展有符
号整型,存在一个对应的\df{扩展无符号整型}。标准和扩展无符号整型统称为\df{无符号
整型}。一个无符号整型与对应的有符号整型具有相同的宽度\nt{N}。无符号类型的可表示
值范围为$0$到$2^N-1$(含);无符号类型的算术作模$2^N$处理。

\begin{note}
  无符号算术不会溢出。有符号算术的溢出产生未定义行为(\ref{expr.pre})。
\end{note}

\paragraph{} % 3
一个无符号整型与对应的有符号整型具有相同的对象表示,值表示和对齐要求
(\ref{basic.align})。对一个有符号整型的每一个值$x$,对应无符号整型模$2^N$同余
的值与其值表示的对应位具有相同的值。\footnote{也就是二补码表示。}

\begin{example}
  无符号整型值$-1$与对应无符号类型最大值具有相同表示。
\end{example}

\paragraph{} % 4
每一个有符号整型的宽度不应该小于表\ref{tab:basic.fundamental.width}中所指定值。
一个有符号或无符号整型的值表示由\nt{N}位组成,其中\nt{N}为对应宽度。对象表示中任
何填充位(\ref{basic.types})的每一个值集为值表示的一种备选表示。

\begin{note}
  填充位具有未指明值,但不能引起陷阱。相反,见\isoc{} 6.2.6.2。
\end{note}

\begin{note}
  有符号和无符号整型满足\isoc{} 5.2.4.2.1中给定的约束。
\end{note}

除以上所指定,有符号和无符号整型的宽度由实现定义。

\begin{table}[!ht]
  \centering
  \caption{最小宽度}
  \begin{tabular}{|ll|}
    \hline
    \tb{类型}         & \tb{最小宽度\nt{N}}                                   \\
    \hline\hline
    \tm{signed char}  & 8                                                     \\
    \tm{short}        & 16                                                    \\
    \tm{int}          & 16                                                    \\
    \tm{long}         & 32                                                    \\
    \tm{long long}    & 64                                                    \\
    \hline
  \end{tabular}
  \label{tab:basic.fundamental.width}
\end{table}

\paragraph{} % 5
一个位宽为\nt{N}的无符号整型的每一个值$x$具有一个唯一的表示
$x = x_0 2^0 + x_1 2^1 + \ldots + x_{N-1} 2^{N-1}$,其中每一个系数$x_i$为$0$或
$1$;这称作$x$的\df{以2为基的表示}。一个有符号整型值的以2为基的表示为对应无符号
的同余值的以2为基的表示。标准有符号整型和标准无符号整型统称为\df{标准整型},且扩
展有符号整型和扩展无符号整型统称为\df{扩展整型}。

\paragraph{} % 6
指定为具有有符号或无符号整型作为其\df{底层类型}的基本类型与其底层类型具有相同的
对象表示,值表示,对齐要求(\ref{basic.align})和可表示值范围。更进一步,每个值
在两种类型中都具有相同表示。

\paragraph{} % 7
类型\tm{char}为一个不同类型,具有实现定义的选择“\tm{signed char}”或“\tm{unsigned
char}”作为其底层类型。类型\tm{char}的值可以表示实现的基本字符集中所有成员的不同
编码。\tm{char},\tm{signed char}和\tm{unsigned char}这三种类型统称为\df{普通字
符类型}。普通字符类型和\tm{char8\_t}统称为\df{窄字符类型}。对于窄字符类型,对象
表示的每一个可能的位模式都表示一个不同值。

\begin{note}
  这对其他类型不成立。
\end{note}

\begin{note}
  一个窄字符类型的位域,窄字符类型的宽度比位域宽度更大的话,则位域具有填充位。见
  \ref{basic.types}。
\end{note}

\paragraph{} % 8
类型\tm{wchar\_t}是一个不同类型,按实现定义以有符号或无符号类型作为其底层类型。
\tm{wchar\_t}类型的值可以表示支持本地环境(\ref{locale})所指定的最大扩展字符集
所有成员的不同编码。

\paragraph{} % 9
类型\tm{char8\_t}表示不同类型,其底层类型为\tm{unsigned char}。类型
\tm{char16\_t}和\tm{char32\_t}表示不同类型,其底层类型分别为\tm{<cstdint>}
(\ref{cstdint.syn})中的\tm{uint\_least16\_t}和\tm{uint\_least32\_t}。

\paragraph{} % 10
类型\tm{bool}为一个不同类型,与实现定义的无符号整型具有相同对象表示,值表示和对
齐要求。类型\tm{bool}的值为\tm{true}和\tm{false}。

\begin{note}
  不存在\tm{signed},\tm{unsigned},\tm{short}或\tm{long bool}类型或值。
\end{note}

\paragraph{} % 11
类型\tm{bool},\tm{char},\tm{wchar\_t},\tm{char8\_t},\tm{char16\_t},
\tm{char32\_t}以及有符号和无符号整型统称为\df{整型}(\df{integral type})。整型
的一个同义词是\df{整数类型}(\df{integer type})。

\begin{note}
  枚举(\ref{dcl.enum})不是整型;但无作用域枚举可以按\ref{conv.prom}提升为整
  型。
\end{note}

\paragraph{} % 12
存在三种\df{符点类型}:\tm{float},\tm{double}和\tm{long double}。\tm{double}类
型提供至少\tm{float}的精度,\tm{long double}提供至少\tm{double}的精度。类型
\tm{float}的值集是类型\tm{double}值集的子集;类型\tm{double}的值集是类型\tm{long
double}值集的子集。符点类型的值表示由实现定义。

\begin{note}
  本文件不对符点操作的精度作要求;见\ref{support.limits}。
\end{note}

整型和符点类型统称为\df{算术类型}。标准库模板\tm{std::numeric\_limits}
(\ref{support.limits})的特例化应该为一个实现指定每一种算术类型的最大和最小值。

\paragraph{} % 13
类型\nt{cv} \tm{void}是一个不完整类型,也不能补充完整;这样的类型具有一个空的值
集。其用作不返回值的函数的返回类型。任何表达式可以显式转换为\nt{cv} \tm{void}类
型(\ref{expr.type.conv},\ref{expr.static.cast},\ref{expr.cast})。一个类型为
\nt{cv} \tm{void}类型的表达式可以用作一个表达式语句(\ref{stmt.expr}),一个逗号
表达式的操作数(\ref{expr.comma}),一个\tm{?:}(\ref{expr.cond})的第二或第三操
作数,\tm{typeid}、\tm{noexcept}或\tm{decltype}的操作数,一个具有返回类型为
\nt{cv} \tm{void}的函数的一条\tm{return}语句(\ref{stmt.return})中的表达式,或
者作为一个到\nt{cv} \tm{void}类型显式转换的操作数。

\paragraph{} % 14
一个\tm{std::nullptr\_t}类型的值是一个零指针常量(\ref{conv.ptr})。这样的值参与
指针和成员指针转换(\ref{conv.ptr},\ref{conv.mem})。
\tm{sizeof(std::nullptr\_t)}应该等于\tm{sizeof(void*)}。

\paragraph{} % 15
本子条款中所述类型称为\df{基本类型}。

\begin{note}
  即使实现定义两个或多个基本类型具有相同值表示,它们仍是不同类型。
\end{note}
