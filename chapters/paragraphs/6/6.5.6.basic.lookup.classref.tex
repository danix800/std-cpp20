% 6.5.6 Class member access [basic.lookup.classref]
\paragraph{} % 1
在一个类成员访问表达式(\ref{expr.ref})中,如果\tm{.}或\tm{->}符号之后直接跟上
一个\nt{identifier}再跟上一个\tm{<},则查询该标识符以确定\tm{<}是一个模板实参列
表的开始还是一个小于运算符。标识符首先在对象表达式(\ref{class.member.lookup})
的类中查询。如果未找到该标识符,则再在整个\nt{postfix-expression}的上下文中查询
并且应该确定一个其特例化为类型的模板。

\paragraph{} % 2
如果一个类成员访问(\ref{expr.ref})中的\nt{id-expression}是一个
\nt{unqualified-id},且对象表达式的类型为类类型\tm{C},则\nt{unqualified-id}在类
\tm{C}的作用域(\ref{class.member.lookup})中查询。

\paragraph{} % 3
如果\nt{unqualified-id}是\tm{\tat} \nt{type-name},则\nt{type-name}在整个
\nt{postfix-expression}的上下文中进行查询。如果对象表达式的类型\tm{T}为类类型
\tm{C},则\nt{type-name}也在类\tm{C}的作用域中查询。至少一个查询应该找到引用
\nt{cv} \tm{T}的名字。

\begin{example}
  \begin{lstlisting}
    struct A { };

    struct B {
      struct A { };
      void f(::A* a);
    };

    void B::f(::A* a) {
      a->~A();                // OK: lookup in *a finds the injected-class-name
    }
  \end{lstlisting}
\end{example}

\paragraph{} % 4
如果一个类成员访问中的\nt{id-expression}是一个形如\par
\qquad\tm{\nt{class-name-or-namespace-name}::...} \par
的\nt{qualified-id},则跟在\tm{.}或\tm{->}之后的
\tm{\nt{class-name-or-namespace-name}}首先在对象表达式
(\ref{class.member.lookup})的类中进行查询,且该名字如果找到了即使用它。否则在
整个\nt{postfix-expression}的上下文中进行查询。

\begin{note}
  见\ref{basic.lookup.qual},描述了\tm{::}之前的名字查询,只能找到类型或命名空间
  名。
\end{note}

\paragraph{} % 5
如果\nt{qualified-id}具有形如\par
\qquad\tm{::\nt{class-name-or-namespace-name}::...} \par
则\tm{\nt{class-name-or-namespace-name}}作为一个\nt{class-name}或
\nt{namespace-name}在全局作用域中查询。

\paragraph{} % 6
如果\nt{nested-name-specifier}包含一个\nt{simple-template-id}
(\ref{temp.names}),则其\nt{template-argument}中的名字在整个
\nt{postfix-expression}所出现在的上下文中查询。

\paragraph{} % 7
如果\nt{id-expression}是一个\nt{conversion-function-id},其
\nt{conversion-type-id}首先在对象表达式的类中查询,且名字如果找到即使用它。否则
在整个\nt{postfix-expression}的上下文中查询。在每一个这种查询中,只考虑表示类型
或特例化为类型的模板。

\begin{example}
  \begin{lstlisting}
    struct A { };
    namespace N {
      struct A {
        void g() { }
        template <class T> operator T();
      };
    }

    int main() {
      N::A a;
      a.operator A(); // calls N::A::operator N::A
    }
  \end{lstlisting}
\end{example}
