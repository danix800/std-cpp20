% 6.5.4.1 General [basic.lookup.qual.general]
\paragraph{} % 1
一个类或命名空间成员或\enumr{}的名字可以在\tm{::}作用域解析运算符
(\ref{expr.prim.id.qual})应用于一个表示其类,命名空间或枚举的
\nt{nested-name-specifier}之后来引用。如果一个\nt{nested-name-specifier}中的
\tm{::}作用域解析运算符之前无\nt{decltype-specifier},则该\tm{::}之前的名字查询
只考虑命名空间,类型和其特例化是类型的模板。如果所找到的名字不表示命名空间或类,
枚举或依赖类型,则程序为\illform{}。

\begin{example}
  \begin{lstlisting}
    class A {
    public:
      static int n;
    };

    int main() {
      int A;
      A::n = 42;        // OK
      A b;              // error: A does not name a type
    }
  \end{lstlisting}
\end{example}

\paragraph{} % 2
\begin{note}
  多重限定名,比如\tm{N1::N2::N3::n},可用于引用嵌套类(\ref{class.nest})成员或
  嵌套命名空间的成员。
\end{note}

\paragraph{} % 3
在一个\nt{declarator-id}是一个\nt{qualified-id}的声明中,用在所声明
\nt{qualified-id}之前的名字在定义命名空间作用域中查询;跟在\nt{qualified-id}之后
的名字在成员的类或命名空间作用域中查询。

\begin{example}
  \begin{lstlisting}
    class X { };
    class C {
      class X { };
      static const int number = 50;
      static X arr[number];
    };
    X C::arr[number];   // error:
                        // equivalent to ::X C::arr[C::number];
                        // and not to C::X C::arr[C::number];
  \end{lstlisting}
\end{example}

\paragraph{} % 4
一个名字前加一元作用域运算符\tm{::}(\ref{expr.prim.id.qual})在所用的翻译单元
中在全局作用域中查询。该名字应该声明于全局命名空间作用域中,或者应该是其声明因
\nt{using-directive}(\ref{namespace.qual})而在全局作用域中可见的名字。\tm{::}
的使用允许一个全局名字即使其标识符被隐藏(\ref{basic.scope.hiding})了仍可以引
用。

\paragraph{} % 5
一个名字前加指定一个枚举类型的\nt{nested-name-specifier}应该表示该枚举的一
个\enumr{}。

\paragraph{} % 6
在一个形如\par
\mbox\qquad\nt{nested-name-specifier\tsub{opt} type-name} \tm{::\tat}
  \nt{type-name}\par
的\nt{qualified-id}中,第二个\nt{type-name}在与第一个相同的作用域中查询。

\begin{example}
  \begin{lstlisting}
    struct C {
      typedef int I;
    };
    typedef int I1, I2;
    extern int* p;
    extern int* q;
    p->C::I::~I();      // I is looked up in the scope of C
    q->I1::~I2();       // I2 is looked up in the scope of the postfix-expression

    struct A {
      ~A();
    };
    typedef A AB;
    int main() {
      AB* p;
      p->AB::~AB();     // explicitly calls the destructor for A
    }
  \end{lstlisting}
\end{example}

\begin{note}
  \ref{basic.lookup.classref}描述名字查询在\tm{.}和\tm{->}运算符之后如何进行。
\end{note}
