% 4.1.2 Abstract machine [intro.abstract]
\paragraph{} % 1
本文件中的语义描述定义了一个参数化的不确定性抽象机。本文件不对合规实现的结构作要
求。特别的,实现不需要复制或模拟该抽象机的结构。而是要求合规实现(仅)模拟下文解
释的抽象机的可观察行为。\footnote{该条文有时称作“如同”规则,因为只要在程序的可观
察行为的范围内结果\df{如同}要求已经遵循,实现即可自由地忽视本文件中的任何要求。
比如,如果能推断出值不会被使用且不会有影响程序可观察行为的副作用产生,则一个实际
的实现不需要求值表达式的一部分。}

\paragraph{} % 2
抽象机的某些方面和操作在本文件中描述为实现定义的(比如\tm{sizeof(int)})。这些构
成抽象机的参数。每一个实现应该包含记录以描述其这些方面的特征与行为。\footnote{该
记录也包括条件支持结构和本地环境特定的行为。见\ref{intro.compliance}。}这样的记
录应该定义对应于该实现的抽象机(下文称为“对应实例”)。

\paragraph{} % 3
抽象机的某些其他的方面和操作在本文件中描述为未指明的(比如,函数调用中实参的求值
顺序(\ref{expr.call}))。在可能的地方本文件定义了一组允许的行为。这些定义了抽
象机的不确定性方面。因此一个抽象机的实例对于一个给定的程序和输入可以具有多于一个
可能的执行。

\paragraph{} % 4
某些其他操作本文件中描述为未定义的(比如,尝试修改常量对象所产生的效果)。

\begin{note}
  本文件不对含有未定义行为的程序施加要求。
\end{note}

\paragraph{} % 5
执行\wellform{}程序的合规实现对于同样的程序和输入应该产生相同的可观察结果作为抽
象机对应实例的一种可能的执行。但如果任何这样的执行含有未定义操作,则本文件不对使
用该输入来执行该程序的实现施加要求(即使是对第一个未定义操作之前的操作)。

\paragraph{} % 6
合规实现的最小要求为:
\begin{enumerate}
  \item 通过易失\glvalue{}的访问严格按照抽象机的规则进行求值。
  \item 在程序终止时,所有写到文件的数据应该等价于根据抽象机的语义对程序的执行可
        能产生的结果之一。
  \item 交互式设备的输入输出动态应该以一种方式进行,即在程序等待输入之前,实际上
        会发出输出提示。 交互式设备的构成由实现定义。
\end{enumerate}
这些统称为程序的\df{可观察行为}。

\begin{note}
抽象机与实际语义间更严格的对应可由实现来定义。
\end{note}
