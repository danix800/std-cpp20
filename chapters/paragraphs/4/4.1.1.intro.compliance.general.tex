% 4.1.1 General [intro.compliance.general]
\paragraph{} % 1
\df{可诊断规则}集合包含本文件中除显式标记为“无需诊断”或者描述为产生“未定义行为”
的所有语法和语义规则。

\paragraph{} % 2
尽管本文件只陈述\cpp{}实现的要求,将这些要求说成是程序或程序的一部分或程序执行的
要求通常会更好理解。这些要求具有以下涵义:
\begin{enumerate}
  \item 如果程序未违反第\ref{lex}条至第\ref{thread}条以及附录\ref{depr}中规则,
        则合规实现应该在描述于附录\ref{implimits}中的资源限度内接受并正确执行
        \footnote{根据所处理的数据,“正确执行”可以包括未定义行为;见第
        \ref{intro.defs}条和\ref{intro.execution}}该程序。
  \item 如果程序违反了任一条可诊断规则或者出现了本文件中描述为“条件支持”的结构而
        实现又不支持该结构,则合规实现应该给出至少一条诊断消息。
  \item 如果程序违反了某一条不需要诊断的规则则本文件不对实现相对该程序作要求。
\end{enumerate}
\begin{note}
  在其他上下文中需要诊断的结构在模板实参推导和替换过程中会区别对待;见
  \ref{temp.deduct}。
\end{note}

\paragraph{} % 3
对于类和类模板,标准库条款规定部分定义。不指定私有成员(\ref{class.access}),但
实现应该根据标准库条款提供它们以补全定义。

\paragraph{} % 4
对于函数,函数模板,对象和值,标准库条款规定其声明。实现应该提供与标准库条款描述
相一致的定义。

\paragraph{} % 5
定义于标准库中的名字具有命名空间作用域(\ref{basic.namespace})。一个\cpp{}翻译
单元(\ref{lex.phases})通过包含合适的标准库头或导入合适的标准库命名头单元
(\ref{using.headers})来获得对名字的访问。

\paragraph{} % 6
标准库中的模板,类,函数和对象具有外部链接(\ref{basic.link})。实现在组合翻译单
元以形成完整\cpp{}程序时(\ref{lex.phases})按需为标准库实体提供定义。

\paragraph{} % 7
定义两种类型的实现:\df{宿主式实现}和\df{自由式实现}。对于宿主式实现,本文件定义
了一组可用标准库。自由式实现指可以在无操作系统支持下执行的实现,具有含某些语言支
持库的实现定义标准库集合(\ref{compliance})。

\paragraph{} % 8
合规实现可以有扩展(包括额外的标准库函数),只要其不改变任何\wellform{}程序的行
为。实现需要按本文件对此类扩展的\illform{}使用给出诊断。但在诊断之后实现应该能够
编译并执行这样的程序。

\paragraph{} % 9
每一个实现应该包含识别其未支持的条件支持结构和定义特定本地环境特征的记录。
\footnote{本文件还定义了实现定义行为;见\ref{intro.abstract}}
