% 10.1 Module units and purviews [module.unit]
\synsym{module-declaration}
  \synprd{\nt{export-keyword\tsub{opt}} \nt{module-keyword} \nt{module-name}
    \nt{module-partition\tsub{opt}}}
    \synprd{\ \ \ \ \nt{attribute-specifier-seq\tsub{opt}} \tm{:}}
\synsym{module-name}
  \synprd{\nt{module-name-qualifier\tsub{opt}} \nt{identifier}}
\synsym{module-partition}
  \synprd{\tm{:} \nt{module-name-qualifier\tsub{opt}} \nt{identifier}}
\synsym{module-name-qualifier}
  \synprd{\nt{identifier} \tm{.}}
  \synprd{\nt{module-name-qualifier} \nt{identifier} \tm{.}}

\paragraph{} % 1
\nt{模块单元}指包含一个\nt{module-declaration}的翻译单元。\nt{命名模块}指具有相
同的\nt{module-name}的模块单元集合。标识符\tm{module}和\tm{import}不应该用作一
个\nt{module-name}或\nt{module-partition}中的\nt{identifier}。所有以包含
\tm{std}后跟零个或多个\nt{digit}的\nt{identifier}开始,或者包含保留标识符
(\ref{lex.name})的\nt{module-name}都是保留的,不应该出现在
\nt{module-declaration}中;无需诊断。如果一个保留\nt{module-name}中的任何
\nt{identifier}是一个保留标识符,则模块名保留用于\cpp{}实现;否则保留用作未来标
准化。可选的\nt{attribute-specifier-seq}用于\nt{module-declaration}。

\paragraph{} % 2
\nt{module-declaration}以\nt{export-keyword}开头的模块单元是一个\nt{模块接口单
元};其他模块单元称为\nt{模块实现单元}。一个命名模块应该包含仅一个不含
\nt{module-partition}的模块接口单元,称为模块的\nt{主模块接口单元};无需诊断。

\paragraph{} % 3
一个\nt{module-declaration}中包含\nt{module-partition}的模块单元称为
\nt{模块分区}。命名模块不应该包含相同的\nt{module-partition}多次。模块接口单元
的所有模块分区应该直接或间接地由主模块接口单元导出(\ref{module.import})。违反
此规则无需诊断。

\begin{note}
  模块分区只能被同一模块中的其他模块单元导入。将模块切分为模块单元对外不可见。
\end{note}

\paragraph{} % 4
\begin{example}

  翻译单元\#1:
  \begin{lstlisting}
  export module A;
  export import :Foo;
  export int baz();
  \end{lstlisting}
  翻译单元\#2:
  \begin{lstlisting}
  export module A::Foo;
  import :Internals;
  export int foo() { return 2 * (bar() + 1); }
  \end{lstlisting}
  翻译单元\#3:
  \begin{lstlisting}
  module A:Internals;
  int bar();
  \end{lstlisting}
  翻译单元\#4:
  \begin{lstlisting}
  module A;
  import :Internals;
  int bar() { return baz() - 10; }
  int baz() { return 30; }
  \end{lstlisting}
  模块\tm{A}包含4个翻译单元:
  \begin{enumerate}
    \item 一个主模块接口单元,
    \item 一个模块分区\tm{A:Foo},这是一个模块接口单元,形成模块\tm{A}接口的一
      部分
    \item 一个模块分区\tm{A:Internals},对模块\tm{A}的外部接口无作用,以及
    \item 一个模块实现单元,提供\tm{bar}和\tm{baz}的实现,不能导出,因为它没有
      模块分区名。
  \end{enumerate}
\end{example}

\paragraph{} % 5
一个\nt{模块单元范围}指从\nt{module-declaration}开始直到翻译单元结束的\nt{token}
序列。命名模块\tm{M}的\nt{范围}指\tm{M}的模块单元范围集合。

\paragraph{} % 6
\nt{全局模块}指所有\nt{global-module-fragment}以及所有不是模块单元的翻译单元
集合。出现在这样的上下文中的声明称为处于全局模块的\nt{范围}中。

\begin{note}
  全局模块没有名字,没有模块接口单元,也不由任何\nt{module-declaration}引入。
\end{note}

\paragraph{} % 7
一个模块要么是命名模块,要么是全局模块。一个声明按如下\nt{附属于}一个模块:
\begin{enumerate}
  \item 如果声明
    \begin{enumerate}
      \item 是一个可替换的全局分配或释放函数
        (\ref{new.delete.single},\ref{new.delete.array}),或者
      \item 是一个具有外部链接的\nt{namespace-definition},或者
      \item 出现在\nt{linkage-specification}中,
    \end{enumerate}
    则它附属于全局模块。
  \item 否则该声明附属于其所在范围的模块中。
\end{enumerate}

\paragraph{} % 8
一个即不包含\nt{export-keyword}也不包含\nt{module-partition}的
\nt{module-declaration}如同使用\nt{module-import-declaration}来隐式导入模块的主
模块接口单元。

\begin{example}

  翻译单元\#1:
  \begin{lstlisting}
  module B:Y;           // does not implicitly import B
  int y();
  \end{lstlisting}
  翻译单元\#2:
  \begin{lstlisting}
  export module B;
  import :Y;            // OK, does not create interface dependency cycle
  int n = y();
  \end{lstlisting}
  翻译单元\#3:
  \begin{lstlisting}
  module B:X1;          // does not implicitly import B
  int &a = n;           // error: n not visible here
  \end{lstlisting}
  翻译单元\#4:
  \begin{lstlisting}
  module B:X2;          // does not implicitly import B
  import B;
  int &b = n;           // OK
  \end{lstlisting}
  翻译单元\#5:
  \begin{lstlisting}
  module B;             // implicitly imports B
  int &c = n;           // OK
  \end{lstlisting}
\end{example}
