% 10.5 Private module fragment [module.private.frag]
\synsym{private-module-fragment}
  \synprd{\nt{module-keyword}\tm{ : private ; }\nt{declaration-seq\tsub{opt}}}

\paragraph{} % 1
一个\nt{private-module-fragment}应该只出现在主模块接口单元中
(\ref{module.unit})。具有\nt{private-module-fragment}的模块单元应该是其模块的
唯一模块单元;无需诊断。

\paragraph{} % 2
\begin{note}
  一个\nt{private-module-fragment}终止一个影响其他翻译单元行为的模块接口单元部
  分。一个\nt{private-module-fragment}允许模块表示为单个翻译单元而不会使模块所
  有内容对导入者可达。\nt{private-module-fragment}的存在会影响:
  \begin{enumerate}
    \item 要求一个导出内联函数定义的点(\ref{dcl.inline}),
    \item 要求一个具有占位符返回类型的导出函数的点(\ref{dcl.spec.auto}),
    \item 声明是否要求不被暴露(\ref{basic.link}),
    \item 内联函数和模块定义必须出现的位置(\ref{basic.def.odr},
      \ref{dcl.inline},\ref{temp.pre}),
    \item 在其之前实例化的模板的实例化上下文,以及
    \item 其内声明的可达性(\ref{module.reach})。
  \end{enumerate}
\end{note}

\paragraph{} % 3
\begin{example}
  \begin{lstlisting}
  export module A;
  export inline void fn_e();  // error: exported inline function fn_e not defined
                              // before private module fragment
  inline void fn_m();         // OK, module-linkage inline function
  static void fn_s();
  export struct X;
  export void g(X *x) {
    fn_s();                   // OK, call to static function in same translation unit
    fn_m();                   // OK, call to module-linkage inline function
  }
  export X *factory();        // OK

  module :private;
  struct X {};                // definition not reachable from importers of A
  X *factory() {
    return new X ();
  }
  void fn_e() {}
  void fn_m() {}
  void fn_s() {}
  \end{lstlisting}
\end{example}
