% 2 Normative references [intro.refs]
\paragraph{}
本文件引用以下文件,其全部或部分内容构成本文件的要求。对于已过时引用,仅所引用的
版本有效。对于未过时引用,所引用文件的最新版本有效(含任何修订)。
\begin{enumerate}
  \item{ISO/IEC 2382,\nt{Information technology --- Vocabulary}}
  \item{ISO 8601:2004,\nt{Data elements and interchange formats - Information
    interchange - Representation of dates nd times}}
  \item{ISO/IEC 9899:2018,\nt{Programming languages --- C}}
  \item{ISO/IEC 9945:2003,\nt{Information Technology --- Portable Operating
    System Interface (POSIX)}\footnote{POSIX \textregistered{}是电气和电子工程师
    协会的注册商标。给出此信息是仅为方便本文件用户,不构成ISO或IEC对该产品的认可
    。}}
  \item{ISO/IEC 10646,\nt{Information technology --- Universal Coded Character
    Set (UCS)}}
  \item{ISO/IEC 10646:2003,\footnote{已被ISO/IEC 10646:2017取代。}
    \nt{Information technology --- Universal Multiple-Octet Coded Character Set
    (UCS)}}
  \item{ISO 80000-2:2009,\nt{Quantities and units --- Part 2: Mathematical
    signs and symbols to be used in the natural sciences and technology}}
  \item{Ecma International, \nt{ECMAScript} \footnote{ECMAScript
    \textregistered{}为Ecma International的注册商标。给出此信息是仅为方便本文件
    用户,不构成ISO或IEC对该产品的认可。}\nt{Language Specification},Standard
    Ecma-262,third edition,1999。}
\end{enumerate}

\paragraph{}
ISO/IEC 9800:2017中第7条所述标准库在下文中称为\nt{\c{}标准库}。\footnote{在第
\ref{support}条到第\ref{thread}条和附录\ref{diff}的限定下,\c{}标准库是\cpp{}标
准的子集。}

\paragraph{}
ISO/IEC 9945:2003中所述的操作系统接口在下文中称为\nt{POSIX}。

\paragraph{}
Standard Ecma-262中所述的ECMAScript语言规范在下文中称为\nt{ECMA-262}。

\paragraph{}
\begin{note}
对ISO/IEC 10646:2003的引用仅用于支持废弃特性(\ref{depr.locale.stdcvt})。
\end{note}
