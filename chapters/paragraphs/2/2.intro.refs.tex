% 2 Normative references [intro.refs]
\paragraph{} % 1
本文件引用以下文件,其全部或部分内容构成本文件的要求。对于已过时引用,仅所引用的
版本有效。对于未过时引用,所引用文件的最新版本有效(含任何修订)。
\begin{enumerate}
  \item ISO/IEC 2382,\df{Information technology --- Vocabulary}
  \item ISO 8601:2004,\df{Data elements and interchange formats - Information
        interchange - Representation of dates nd times}
  \item ISO/IEC 9899:2018,\df{Programming languages --- C}
  \item ISO/IEC 9945:2003,\df{Information Technology --- Portable Operating
        System Interface (POSIX)}\footnote{POSIX \textregistered{}是电气和电子工
        程师协会的注册商标。给出此信息是仅为方便本文件用户,不构成ISO或IEC对该产
        品的认可。}
  \item ISO/IEC 10646,\df{Information technology --- Universal Coded Character
        Set (UCS)}
  \item ISO/IEC 10646:2003,\footnote{已被ISO/IEC 10646:2017取代。}
        \df{Information technology --- Universal Multiple-Octet Coded Character
        Set (UCS)}
  \item ISO 80000-2:2009,\df{Quantities and units --- Part 2: Mathematical
        signs and symbols to be used in the natural sciences and technology}
  \item Ecma International, \df{ECMAScript} \footnote{ECMAScript
        \textregistered{}为Ecma International的注册商标。给出此信息是仅为方便本
        文件用户,不构成ISO或IEC对该产品的认可。}\df{Language Specification},
        Standard Ecma-262,third edition,1999。
\end{enumerate}

\paragraph{} % 2
ISO/IEC 9800:2017中第7条所述标准库在下文中称为\df{\c{}标准库}。\footnote{在第
\ref{support}条到第\ref{thread}条和附录\ref{diff}的限定下,\c{}标准库是\cpp{}标
准库的子集。}

\paragraph{} % 3
ISO/IEC 9945:2003中所述的操作系统接口在下文中称为\df{POSIX}。

\paragraph{} % 4
Standard Ecma-262中所述的ECMAScript语言规范在下文中称为\df{ECMA-262}。

\paragraph{} % 5
\begin{note}
  对ISO/IEC 10646:2003的引用仅用于支持废弃特性(\ref{depr.locale.stdcvt})。
\end{note}
