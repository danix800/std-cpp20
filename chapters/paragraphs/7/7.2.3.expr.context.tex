% 7.2.3 Context dependence [expr.context]
\paragraph{} % 1
某些上下文中会出现\df{未求值操作数}(\ref{expr.prim.req},\ref{expr.typeid},
\ref{expr.sizeof},\ref{expr.unary.noexcept},\ref{dcl.type.decltype},
\ref{temp.pre},\ref{temp.concept})。未求值操作数不会进行求值。

\begin{note}
  在一个未求值操作数中可以指定一个非静态类成员(\ref{expr.prim.id}),而指定对象
  或函数本身不要求提供定义(\ref{basic.def.odr})。一个未求值操作数认为是一个全
  表达式(\ref{intro.execution})。
\end{note}

\paragraph{} % 2
在某些上下文中一个表达式仅为其副作用而出现。这样的表达式称为\df{弃值表达式}。不
应用数组到指针(\ref{conv.array})和函数到指针(\ref{conv.func})标准转换。当且
仅当表达式是一个易失限定类型的\glvalue{}且为以下之一时才应用\lvalue{}到\rvalue{}
转换(\ref{conv.lval}):
\begin{enumerate}
  \item \tm{(} \nt{expression} \tm{)},其中\nt{expression}为这些表达式其中之一,
  \item \nt{id-expression}(\ref{expr.prim.id}),
  \item 下标(\ref{expr.sub}),
  \item 类成员访问(\ref{expr.ref}),
  \item 间接取值(\ref{expr.unary.op}),
  \item 成员指针操作(\ref{expr.mptr.oper}),
  \item 条件表达式(\ref{expr.cond}),其中第二和第三操作数均为这些表达式其中之
        一,或者
  \item 逗号表达式(\ref{expr.comma}),其中右操作数为这些表达式其中之一。
\end{enumerate}

\begin{note}
  使用重载运算符引用函数调用;以上仅覆盖具有内置语义的运算符。
\end{note}

如果(可能转换的)表达式是一个\prvalue{},则应用临时\mat{}转换
(\ref{conv.rval})。

\begin{note}
  如果表达式是一个类类型\lvalue{},其必须具有一个易失拷贝构造函数以初始化该临时
  对象,该对象是\lvalue{}到\rvalue{}转换的结果对象。
\end{note}

对该\glvalue{}表达式进行求值,其值被丢弃。
