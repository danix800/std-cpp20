% 7.5.4.1 General [expr.prim.id.general]
\synsym{id-expression}
  \synprd{\nt{unqualified-id}}
  \synprd[]{\nt{qualified-id}}

\paragraph{} % 1
一个\nt{id-expression}是一个\nt{primary-expression}的限制形式。

\begin{note}
  一个\nt{id-expression}可以出现在\tm{.}和\tm{->}运算符(\ref{expr.ref})之后。
\end{note}

\paragraph{} % 2
一个表示类的非静态数据成员或非静态成员函数的\nt{id-expression}只能用于:
\begin{enumerate}
  \item 作为类成员访问(\ref{expr.ref})的一部分,其中对象表达式引用该成员的类
        \footnote{当对象表达式是一个隐式的\tm{(*this)}
        (\ref{class.mfct.non-static})时也适用。}或派生自该类的类,或者
  \item 形成一个成员指针(\ref{expr.unary.op}),或者
  \item 如果该\nt{id-expression}表示一个非静态数据成员且其出现在一个未求值操作数
        中。

\begin{example}
  \begin{lstlisting}
    struct S {
    int m;
    };
    int i = sizeof(S::m); // OK
    int j = sizeof(S::m + 42); // OK
  \end{lstlisting}
\end{example}
\end{enumerate}

\paragraph{} % 3
一个表示中间函数(\ref{dcl.constexpr})的潜在求值\nt{id-expression}应该仅出现
\begin{enumerate}
  \item 作为一个中间调用的子表达式,或者
  \item 在一个中间函数上下文中(\ref{expr.const})。
\end{enumerate}

\paragraph{} % 4
对一个表示重载集合的\nt{id-expression},应用重载解析来选择一个唯一的函数
(\ref{over.match},\ref{over.over})。

\begin{note}
  一个程序不能引用具有一个其\nt{constraint-expression}不能满足的结尾
  \nt{requires-clause}的函数,因这种函数不会被重载解析选中。

\begin{example}
  \begin{lstlisting}
    template<typename T> struct A {
      static void f(int) requires false;
    }

    void g() {
      A<int>::f(0);                       // error: cannot call f
      void (*p1)(int) = A<int>::f;        // error: cannot take the address of f
      decltype(A<int>::f)* p2 = nullptr;  // error: the type decltype(A<int>::f) is invalid
    }
  \end{lstlisting}
  在每一种情况下\tm{f}的约束都不能满足。在\tm{p2}的声明中即使\tm{f}是一个未求值
  操作也要求满足这些约束(\ref{expr.prop})。\end{example}

\end{note}
