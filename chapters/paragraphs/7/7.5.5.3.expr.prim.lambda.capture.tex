% 7.5.5.3 Captures [expr.prim.lambda.capture]
\synsym{lambda-capture}
  \synprd{\nt{capture-default}}
  \synprd{\nt{capture-list}}
  \synprd{\nt{capture-default} \tm{,} \nt{capture-list}}
\synsym{capture-default}
  \synprd{\tm{\&}}
  \synprd{\tm{=}}
\synsym{capture-list}
  \synprd{\nt{capture}}
  \synprd{\nt{capture-list} \tm{,} \nt{capture}}
\synsym{capture}
  \synprd{\nt{simple-capture}}
  \synprd{\nt{init-capture}}
\synsym{simple-capture}
  \synprd{\nt{identifier} \tm{...}\nt{\tsub{opt}}}
  \synprd{\tm{\&} \nt{identifier} \tm{...}\nt{\tsub{opt}}}
  \synprd{\tm{this}}
  \synprd{\tm{* this}}
\synsym{init-capture}
  \synprd{\tm{...}\nt{\tsub{opt}} \nt{identifier} \nt{initializer}}
  \synprd{\tm{\& ...}\nt{\tsub{opt}} \nt{identifier} \nt{initializer}}

\paragraph{} % 1
\nt{lambda-expression}的函数体可以按如下所述的捕获来访问包含作用域中的实体。

\paragraph{} % 2
如果\nt{lambda-capture}包含一个\nt{capture-default}为\tm{\&},则该
\nt{lambda-capture}的\nt{simple-capture}中标识符不应该以\tm{\&}开头。如果
\nt{lambda-capture}包含一个\nt{capture-default}为\tm{=},则该
\nt{lambda-capture}的每一个\nt{simple-capture}应形如
“\tm{\&} \nt{identifier} \tm{...}\nt{\tsub{opt}}”、“\tm{this}”或“\tm{* this}”。

\begin{note}
  为与ISO \cpp{} 2014兼容,形式\tm{[\&,this]}是多余的,但可接受。
\end{note}

忽略\nt{init-capture}中的\nt{initializer},标识符或\tm{this}在
\nt{lambda-capture}不应该出现多于一次。

\begin{example}
  \begin{lstlisting}
  struct S2 { void f(int i); };
  void S2::f(int i) {
    [&, i]{ }         // OK
    [&, this, i]{ };  // OK, equivalent to [&, i]
    [&, &i]{ };       // error: i preceded by & when & is the default
    [=, *this]{ };    // OK
    [=, this]{ };     // OK, equivalent to [=]
    [i, i]{ };        // error: i repeated
    [this, *this]{ }; // error: this appears twice
  }
  \end{lstlisting}
\end{example}

\paragraph{} % 3
一个\nt{lambda-expression}在其\nt{lambda-introducer}中不应该具有
\nt{capture-default}或\nt{simple-capture},除非其最内层包含作用域是块作用域
(\ref{basic.scope.block})或它出现在一个缺省成员初始化中,而其最内层作用域是对
应的类作用域(\ref{basic.scope.class})。

\paragraph{} % 4
\nt{simple-capture}中的\nt{identifier}使用常规的未限定名查询
(\ref{basic.lookup.unqual})规则进行查询;每个这种查询都应该找到一个局部实体。
\nt{simple-capture} \tm{this}和\tm{* this}表示局部实体\tm{*this}。由
\nt{simple-capture}表示的实体称为\nt{显式捕获}。

\paragraph{} % 5
如果\nt{simple-capture}中的\nt{identifier}用作\nt{lambda-declarator}的
\nt{parameter-declaration-clause}的形参\nt{declarator-id}则程序为非法。

\begin{example}
  \begin{lstlisting}
  void f() {
    int x = 0;
    auto g = [x](int x) { return 0; };  // error: parameter and simple-clause have the same name
  }
  \end{lstlisting}
\end{example}

\paragraph{} % 6
不带省略号的\nt{init-capture}如同声明并显式捕获了形如
“\tm{auto} \nt{init-capture} \tm{;}”的变量,其声明区为\nt{lambda-expression}的
\nt{compound-statement},除非:
\begin{enumerate}
  \item 如果捕获为拷贝(见下文),则为捕获声明的非静态数据成员和变量被当作引用
        相同对象的两种不同方式,具有非静态数据成员的生命期,不进行额外的拷贝和
        析构,并且
  \item 如果捕获为引用,变量生命期在闭包对象生命期结束时结束。
\end{enumerate}

\begin{note}
  这允许形如“\tm{x = std::move(x)}”的\nt{init-capture};第二个“\tm{x}”必须绑
  定到包含上下文中的声明。
\end{note}

\begin{example}
  \begin{lstlisting}
  int x = 4;
  auto y = [&r = x, x = x+1]()->int {
              r += 2;
              return x+2;
            }();              // Updates ::x to 6, and initializes y to 7.

  auto z = [a = 42](int a) { return 1; }; // error: parameter and local variable have the same name
  \end{lstlisting}
\end{example}

\paragraph{} % 7
为实现$\lambda$捕获,一个表达式按如下潜在引用局部实现:
\begin{enumerate}
  \item 确定局部实体的\nt{id-expression}潜在引用该实现;确定一个或多个非静态类
        成员,但不形成成员指针(\ref{expr.unary.op})的\nt{id-expression}潜在引
        用\tm{*this}。
        \begin{note}
          即使重载解析为该\nt{id-expression}选择了静态成员函数也是如此。
        \end{note}
  \item \tm{this}表达式潜在引用\tm{*this}。
  \item \nt{lambda-expression}潜在引用其\nt{simple-capture}确定的局部实体。
\end{enumerate}
如果表达式在一个声明区内潜在引用一个局部实体,在声明内该实体是odr可用的,并且如
果忽略任何包含的\tm{typeid}表达式(\ref{expr.typeid})的副作用则表达式将是被潜
在求值的,则实体称为被中间的每一个带有关联\nt{capture-default},未显式捕获的
\nt{lambda-expression}\nt{隐式捕获}。当\nt{capture-default}为\tm{=}时\tm{*this}
的隐式捕获将被废弃;见\ref{depr.capture.this}。

\begin{example}
  \begin{lstlisting}
  void f(int, const int (&)[2] = {});   // #1
  void f(const int&, const int (&)[1]); // #2
  void test() {
    const int x = 17;
    auto g = [](auto a) {
      f(x);               // OK: calls #1, does not capture x
    };

    auto g1 = [=](auto a) {
      f(x);               // OK: calls #1, captures x
    };

    auto g2 = [=](auto a) {
      int selector[sizeof(a) == 1 ? 1 : 2]{};
      f(x, selector);     // OK: captures x, can call #1 or #2
    };

    auto g3 = [=](auto a) {
      typeid(a + x);      // captures x regardless of whether a + x is an unevaluated operand
    };
  }
  \end{lstlisting}
在\tm{g1}中,实现可以优化掉\tm{x}的捕获,因其不是odr使用的。
\end{example}

\begin{note}
  捕获的实体集从语法上决定,且实体是隐式捕获的,即使表达式表示一个丢弃语句
  (\ref{stmt.if})中的局部实体。

  \newpage
  \begin{example}
    \begin{lstlisting}
    template<bool B>
    void f(int n) {
      [=](auto a) {
        if constexpr (B && sizeof(a) > 4) {
          (void)n;                  // captures n regardless of the value of B and sizeof(int)
        }
      }(0);
    }
    \end{lstlisting}
  \end{example}

\end{note}

\paragraph{} % 8
如果被显式或隐式捕获,则称实体是被\nt{捕获}的。由\nt{lambda-expression}捕获的实
体在包含\nt{lambda-expression}的作用域中是odr使用的(\ref{basic.def.odr})。

\begin{note}
  其结果是,如果一个\nt{lambda-expression}显式捕获一个不是odr使用的实体,则程序
  是非法的(\ref{basic.def.odr})。
\end{note}

\begin{example}
  \begin{lstlisting}
  void f1(int i) {
    int const N = 20;
    auto m1 = [=]{
      int const M = 30;
      auto m2 = [i]{
        int x[N][M];      // OK: N and M are not odr-used
        x[0][0] = i;      // OK: i is explicitly captured by m2 and implicitly captured by m1
      };
    };
    struct s1 {
      int f;
      void work(int n) {
        int m = n*n;
        int j = 40;
        auto m3 = [this,m] {
          auto m4 = [&,j] { // error: j not odr-usable due to intervening lambda m3
            int x = n;      // error: n is odr-used but not odr-usable due to intervening lambda m3
            x += m;         // OK: m implicitly captured by m4 and explicitly captured by m3
            x += i;         // error: i is odr-used but not odr-usable
                            // due to intervening function and class scopes
            x += f;         // OK: this captured implicitly by m4 and explicitly by m3
          };
        };
      }
    };
  }

  struct s2 {
    double ohseven = .007;
    auto f() {
      return [this] {
        return [*this] {
          return ohseven;   // OK
        };
      }();
    }
    auto g() {
      return [] {
        return [*this] { }; // error: *this not captured by outer lambda-expression
      }();
    }
  };
  \end{lstlisting}
\end{example}

\newpage

\paragraph{} % 9
\begin{note}
  因局部实体在缺省实参(\ref{basic.def.odr})中不是odr可用的,一个出现在缺省实
  参中的\nt{lambda-expression}不能被任何实体隐式或显式捕获。如果其
  \nt{initializer}中的任何全表达式满足出现在缺省实参(\ref{dcl.fct.default})中
  表达式的约束则这样的\nt{lambda-expression}仍可以有一个\nt{init-capture}。
\end{note}

\begin{example}
  \begin{lstlisting}
  void f2() {
    int i = 1;
    void g1(int = ([i]{ return i; })());        // error
    void g2(int = ([i]{ return 0; })());        // error
    void g3(int = ([=]{ return i; })());        // error
    void g4(int = ([=]{ return i; })());        // OK
    void g5(int = ([]{ return sizeof i; })());  // OK
    void g6(int = ([x=1]{ return x; })());      // OK
    void g7(int = ([x=i]{ return x; })());      // error
  }
  \end{lstlisting}
\end{example}

\paragraph{} % 10
一个实体是被\nt{拷贝捕获}的,如果
\begin{enumerate}
  \item 它是被隐式捕获的,\nt{capture-default}为\tm{=},且所捕获实体不是
        \tm{*this},或者
  \item 它是被不是\tm{this}、\tm{\&} \nt{identifier}或
        \tm{\&} \nt{identifier initializer}之一的捕获所显式捕获的。
\end{enumerate}

对每一个拷贝捕获的实体,在其闭包类型中声明一个无名非静态数据成员。这些成员的声
明顺序未指明。如果实体是一个对象的引用则声明成员的类型为引用类型,如果实体是一
个函数的引用,则声明成员为所引用函数类型的左值引用,否则为对应所捕获类型。一个
匿名联合不应该被拷贝捕获。

\paragraph{} % 11
一个\nt{lambda-expression}的\nt{compound-statement}中的每一个是拷贝捕获实体odr
使用(\ref{basic.def.odr})的\nt{id-expression},被转换为闭包类型的对应无名数据
成员的访问。

\begin{note}
  不是odr使用的\nt{id-expression}引用原实体,不会引用闭包类型的成员。但这样的
  \nt{id-expression}仍能引起实体的隐式捕获。
\end{note}

如果\tm{*this}是拷贝捕获,每一个odr使用的\tm{*this}表达式被转换为引用闭包类型的
对应无名数据成员。

\begin{example}
  \begin{lstlisting}
  void f(const int*);
  void g() {
    const int N = 10;
    [=] {
      int arr[N];       // OK: not an odr-use, refers to automatic variable
      f(&N);            // OK: causes N to be captured; &N points to
                        // the corresponding member of the closure type
    };
  }
  \end{lstlisting}
\end{example}

\paragraph{} % 12
如果被隐式或显式捕获但不是拷贝捕获则实体是\nt{引用捕获}。未指明是否为引用捕获实
体在闭包类型内声明额外 的无名非静态数据成员。如果声明,则这样的非静态数据成员应
该为字面类型。

\begin{example}
  \begin{lstlisting}
  // The inner closure type must be a literal type regardless of how reference captures are represented.
  static_assert([](int n) { return [&n] { return ++n; }(); }(3) == 4);
  \end{lstlisting}
\end{example}

位域或匿名联合的成员不应该被引用捕获。

\paragraph{} % 13
一个\nt{lambda-expression}的\nt{compound-statement}中引用捕获的引用的odr使用的
\nt{id-expression},引用的是引用捕获所绑定的实体,不是所捕获的引用。

\begin{note}
  这样的捕获的有效性由所引用的对象生命期确定,而不是由引用本身的生命期来确定。
\end{note}

\begin{example}
  \begin{lstlisting}
  auto h(int &r) {
    return [&] {
      ++r;              // Valid after h returns if the lifetime of the
                        // object to which r is bound has not ended
    };
  }
  \end{lstlisting}
\end{example}

\paragraph{} % 14
如果\nt{lambda-expression} \tm{m2}捕获一个实体,且该实体被一个直接包含的
\nt{lambda-expression} \tm{m1}所捕获,则\tm{m2}的捕获按如下方式进行变换:
\begin{enumerate}
  \item 如果\tm{m1}对实体拷贝捕获,则\tm{m2}捕获\tm{m1}的闭包类型对应的非静态数
        据成员;
  \item 如果\tm{m1}对实体引用捕获,则\tm{m2}捕获\tm{m1}所捕获的相同实体。
\end{enumerate}

\begin{example}
  \begin{lstlisting}
  int a = 1, b = 1, c = 1;
  auto m1 = [a, &b, &c]() mutable {
    auto m2 = [a, b, &c]() mutable {
      std::cout << a << b << c;
      a = 4; b = 4; c = 4;
    };
    a = 3; b = 3; c = 3;
    m2();
  };
  a = 2; b = 2; c = 2;
  m1();
  std::cout << a << b << c;
  \end{lstlisting}
\end{example}

\paragraph{} % 15
当\nt{lambda-expression}被求值时,拷贝引用的实体用于直接初始化产生的闭包对象中
每一个非静态数据成员,且对应于\nt{init-capture}的非静态数据成员初始化为对应
\nt{initializer}所指定的值(可能拷贝初始化或直接初始化)。(对数组成员,数组元
素按下标升序进行直接初始化。)这些初始化按非(未指明)静态数据成员声明顺序进行。

\begin{note}
  这确保了析构按构造的相反顺序进行。
\end{note}

\paragraph{} % 16
\begin{note}
  如果非引用实体被隐式或显式引用捕获,则在实体生命期结束后调用对应
  \nt{lambda-expression}的函数调用算子很可能产生未定义行为。
\end{note}

\paragraph{} % 17
一个包含省略号的\nt{simple-capture}是一个包展开(\ref{temp.variadic})。一个包
含省略号的\nt{init-capture}是一个引入\nt{init-capture}包(\ref{temp.variadic})
的包展开,其声明区为\nt{lambda-expression}的\nt{compound-statement}。

\begin{example}
  \begin{lstlisting}
  template<class... Args>
  void f(Args... args) {
    auto lm = [&, args...] { return g(args...); };
    lm();

    auto lm2 = [...xs=std::move(args)] { return g(xs...); };
    lm2();
  }
  \end{lstlisting}
\end{example}
