% 7.6.1.7 Dynamic cast [expr.dynamic.cast]
\paragraph{} % 1
表达式\tm{dynamic_cast<T>}的结果为转换表达式\tm{v}到类型\tm{T}的结果。\tm{T}应
该是一个完整类类型的指针或引用,或者是``指向\nt{cv} \tm{void}的指针''。
\tm{dynamic_cast}不应该去掉常属性(\ref{expr.const.cast})。

\paragraph{} % 2
如果\tm{T}为指针类型,\tm{v}应该是一个完整类类型指针的\prvalue{},结果是一个
\tm{T}类型的\prvalue{}。如果\tm{T}是一个\lvalue{}引用类型,\tm{v}应该是一个完整
类类型的\lvalue{},结果是一个\tm{T}所引用类型的\lvalue{}。如果\tm{T}是一个
\rvalue{}引用类型,\tm{v}应该是一个具有完整类类型的\glvalue{},结果是一个\tm{T}
所引用类型的\xvalue{}。

\paragraph{} % 3
如果\tm{v}的类型与\tm{T}相同(忽略cv限定),则结果为(按需转换的)\tm{v}。

\paragraph{} % 4
如果\tm{T}是``指向\nt{cv1} \tm{B}的指针''且\tm{v}的类型为
``指向\nt{cv2} \tm{D}的指针'',其中\tm{B}是\tm{D}的基类,则结果是一个由\tm{v}
所指向\tm{D}对象的唯一子对象\tm{B}的指针,或者如果\tm{v}是空指针的话,则结果为
空指针。类似的,如果\tm{T}是``\nt{cv1} \tm{B}的引用''且\tm{v}具有类型\nt{cv2}
\tm{D},使得\tm{B}是\tm{D}的基类,则结果为由\tm{v}所引用的\tm{D}对象的唯一子对
象\tm{B}。\footnote{\tm{v}所指向或引用的最终派生对象(\ref{intro.object})可以
包含其他 \tm{B}对象作为基类,但这些会被忽略。}在指针和引用这两种情形下,如果
\tm{B}是\tm{D}的不可访问或有歧义基类,则程序为\illform{}。

\begin{example}
  \begin{lstlisting}
  struct B { };
  struct D : B { };
  void foo(D* dp) {
    B*  bp = dynamic_cast<B*>(dp);    // equivalent to B* bp = dp;
  }
  \end{lstlisting}
\end{example}

\paragraph{} % 5
否则,\tm{v}应该是一个多态类型(\ref{class.virtual})的指针或\glvalue{}。

\paragraph{} % 6
如果\tm{v}是一个空指针值,则结果为空指针值。

\paragraph{} % 7
如果\tm{T}是``指向\nt{cv} \tm{void}的指针'',则结果为\tm{v}所指向的最终派生对象
的指针。否则进行运行时检查来确定\tm{v}所指向或引用的对象是否能转换成\tm{T}所指
向或引用的类型。

\paragraph{} % 8
如果\tm{C}是\tm{T}所指向或引用的类类型,则运行时检查逻辑上如下执行:
\begin{enumerate}
  \item 如果在\tm{v}所指向(引用)的最终派生对象中,\tm{v}指向(引用)一个
        \tm{C}对象的公有基类子对象,且仅当一个\tm{C}类型对象派生自\tm{v}所指向
        (引用)的子对象,则结果指向(引用)该\tm{C}对象。
  \item 否则,如果\tm{v}指向(引用)最终派生对象的一个公有基类子对象,且最终派
        生对象的类型为无歧义公有\tm{C}类型基类,则结果指向(引用)最终派生对象
        的\tm{C}子对象。
  \item 否则运行时检查失败。
\end{enumerate}

\paragraph{} % 9
到指针类型的失败转换的值为要求的结果类型的空指针值。失败的引用类型转换抛出一个
该类型的异常(\ref{except.throw}),匹配一个\tm{std::bad_cast}类型的处理程序
(\ref{except.handle})。

\begin{example}
  \begin{lstlisting}
  class A { virtual void f(); };
  class B { virtual void g(); };
  class D : public virtual A, private B { };
  void g() {
    D d;
    B* bp = (B*)&d;                 // cast needed to break protection
    A* ap = &d;                     // public derivation, no cast needed
    D& dr = dynamic_cast<D&>(*bp);  // fails
    ap = dynamic_cast<A*>(bp);      // fails
    bp = dynamic_cast<B*>(ap);      // fails
    ap = dynamic_cast<A*>(&d);      // succeeds
    bp = dynamic_cast<B*>(&d);      // ill-formed (not a runtime check)
  }

  class E : public D, public B { };
  class F : public E, public D { };
  void h() {
    F f;
    A* ap = &f;                     // succeeds: finds unique A
    D* dp = dynamic_cast<D*>(ap);   // fails: yields null; f has two D subobjects
    E* ep = (E*)ap;                 // error: cast from virtual base
    E* ep1 = dynamic_cast<E*>(ap);  // succeeds
  }
  \end{lstlisting}
\end{example}

\begin{note}
  子章节\ref{class.cdtor}描述应用\tm{dynamic_cast}到正在构建或销毁对象时的行为。
\end{note}
