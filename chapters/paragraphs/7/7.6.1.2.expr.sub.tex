% 7.6.1.2 Subscripting [expr.sub]
\paragraph{} % 1
后跟方括号中表达式的后续表达式是一个后缀表达式。其中一个表达式应该是
“\tm{T}数组”类型的广义左值或“\tm{T}的指针”类型的纯右值,且另一个应该是对象类型
的纯右值。\footnote{即使下标运算符用于\tm{\&x[0]}这种习惯用法中也成立。}表达式
\tm{E1[E2]}(定义上)等价于\tm{*((E1)+(E2))},除了在数组操作数的情况下,如果该
操作数是一个左值,其结果为左值,否则为即期值。表达式\tm{E1}前序于表达式\tm{E2}。

\paragraph{} % 2
\begin{note}
  作为下标表达式的\nt{expr-or-braced-init-list}出现的逗号表达式
  (\ref{expr.comma})已被废弃;见\ref{depr.comma.subscript}。
\end{note}

\paragraph{} % 3
\begin{note}
  尽管看起来是非对称的,除序列点外,下标是一个可交换操作。见\ref{expr.unary}和
  \ref{expr.add}关于\tm{*}和\tm{+}以及\ref{dcl.array}关于数组类型细节。
\end{note}

\paragraph{} % 4
\nt{braced-init-list}不应该用于内置下标运算符。
