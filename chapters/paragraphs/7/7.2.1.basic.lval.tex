% 7.2.1 Value category [basic.lval]
\paragraph{} % 1
表达式按图\ref{fig:basic.lval}进行分类。

\begin{figure}[htpb]
  \centering
  \includegraphics[width=0.4\textwidth]{figure/valuecategories.pdf}
  \caption{表达式分类 [fig:basic.lval]}
  \label{fig:basic.lval}
\end{figure}

\begin{enumerate}
  \item 一个\glvalue{}是一个表达式,其求值确定对象,位域或函数的身份。
  \item 一个\prvalue{}是一个表达式,按其所出现的上下文所指定,其求值初始化一个对
        象,位域或计算一个运算符的操作数的值,或是一个具有\nt{cv} \tm{void}类型
        的表达式。
  \item 一个\xvalue{}是一个\glvalue{},表示一个对象或位域,其资源可被重用(通常
        因其接近其生命期终止时)。
  \item 一个\lvalue{}是一个非\xvalue{}的\glvalue{}。
  \item 一个\rvalue{}是一个\prvalue{}或一个\xvalue{}。
\end{enumerate}

\paragraph{} % 2
每一个表达式属于该基本分类当中的一类:\lvalue{},\xvalue{}或\prvalue{}。表达式的
该属性称为其\df{值范畴}。

\begin{note}
  \ref{expr.compound}中关于每一个内置运算符的讨论表明其所产生值范畴以及其所期望
  的操作数的值范畴。比如,内置赋值运算符期望左操作数是一个\lvalue{},右操作数是
  一个\prvalue{},并产生一个\lvalue{}作为结果。自定义运算符是函数,且其所期望和
  所产生的值范畴由其形参和返回类型所确定。
\end{note}

\paragraph{} % 3
\begin{note}
  在历史上,\lvalue{}和\rvalue{}因其出现在赋值的左侧和右侧而得名(尽管这一般不再
  正确);\glvalue{}指“广义的”(generalized)\lvalue{},\prvalue{}指“纯”(pure)
  \rvalue{},而\xvalue{}指“即将到期的”(eXpiring)\lvalue{}。尽管如此命名,这些
  术语实际是对表达式的分类,而不是值。
\end{note}

\paragraph{} % 4
\begin{note}
  一个表达式是一个\xvalue{},如果它是:
  \begin{enumerate}
    \item 调用一个函数的结果,无论隐式还是显式,其返回类型是一个对象类型的
          \rvalue{}引用(\ref{expr.call}),
    \item 一个对象类型的\rvalue{}引用的转换(\ref{expr.type.conv},
          \ref{expr.dynamic.cast},\ref{expr.static.cast},
          \ref{expr.reinterpret.cast},\ref{expr.const.cast},
          \ref{expr.cast}),
    \item 一个\xvalue{}数组操作数的下标操作(\ref{expr.sub}),
    \item 表示一个非引用类型的非静态数据成员的类成员访问表达式,其中对象表达式是
          一个\xvalue{}(\ref{expr.ref}),或者
    \item 一个\tm{.*}成员指针表达式,其中第一个操作数是一个\xvalue{},第二个操作
          数是一个数据成员指针(\ref{expr.mptr.oper})。
  \end{enumerate}
  一般地,该规则的效果是命名\rvalue{}引用当作\lvalue{},而对象的未命名\rvalue{}
  引用当作是\xvalue{};函数的\rvalue{}引用无论是否命名都当作是\lvalue{}。
\end{note}

\begin{example}
  \begin{lstlisting}
    struct A {
      int m;
    };
    A&& operator+(A, A);
    A&& f();

    A a;
    A&& ar = static_cast<A&&>(a);
  \end{lstlisting}
  表达式\tm{f()},\tm{f().m},\tm{static\_cast<A\&\&>(a)},和\tm{a + a}是
  \xvalue{}。表达式\tm{ar}是\lvalue{}。
\end{example}

\paragraph{} % 5
\glvalue{}的\df{结果}是由表达式所表示的实体。\prvalue{}的\df{结果}是该表达式存储
到其上下文中的值;一个\nt{cv} \tm{void}类型的\prvalue{}无结果。一个结果为\nt{V}
的\prvalue{}有时称为具有或确定值\nt{V}。一个\prvalue{}的\df{结果对象}为该
\prvalue{}所初始化的对象;一个用于计算内置运算符操作数的值的非弃值\prvalue{}或者
一个具有\nt{cv} \tm{void}类型的\prvalue{}无结果对象。

\begin{note}
  除非当\prvalue{}是一个\nt{decltype-specifier}的操作数,一个类或数组类型
  \prvalue{}总是具有结果对象。对于一个具有除\nt{cv} \tm{void}类型之外的弃值
  \prvalue{}会\mat{}一个临时对象;见\ref{expr.context}。
\end{note}

\paragraph{} % 6
当一个\glvalue{}作为一个期望\prvalue{}的运算符操作数出现时,应用\lvalue{}到
\rvalue{}(\ref{conv.lval}),数组到指针(\ref{conv.array})或函数到指针
(\ref{conv.func})标准转换以转换表达式为一个\prvalue{}。

\begin{note}
  尝试绑定一个\rvalue{}引用到一个\lvalue{}不是这种上下文;见\ref{dcl.init.ref}。
\end{note}

\begin{note}
  当表达式转换到一个\prvalue{}时,因cv限定符从非类类型表达式的类型中移除,一个
  \tm{const int}类型的\lvalue{}可被用于比如需要\tm{int}类型\prvalue{}的地方。
\end{note}

\begin{note}
  不存在\prvalue{}位域;如果位域被转换成一个\prvalue{}(\ref{conv.lval}),则创
  建一个位域类型的\prvalue{},随后可被提升(\ref{conv.prom})。
\end{note}

\paragraph{} % 7
当一个\prvalue{}作为需要\glvalue{}的运算符的操作数出现时,应用临时\mat{}转换
(\ref{conv.rval})以转换表达式为一个\xvalue{}。

\paragraph{} % 8
\ref{dcl.init.ref}中关于引用初始化的讨论和\ref{class.temporary}中关于临时对象的
讨论指明其他重要上下文中\lvalue{}和\rvalue{}的行为。

\paragraph{} % 9
除非另有指明(\ref{dcl.type.decltype}),一个\prvalue{}应该总是具有完整类型或
\tm{void}类型;如果它具有类类型或(可能多维)类类型数组,则该类不应该是抽象类
(\ref{class.abstract})。一个\glvalue{}不应该具有\nt{cv} \tm{void}类型。

\begin{note}
  一个\glvalue{}可以具有完整或不完整非\tm{void}类型。类或数组\prvalue{}可以具有
  cv限定类型;其他\prvalue{}总是具有cv非限定类型。见\ref{expr.type}。
\end{note}

\paragraph{} % 10
一个\lvalue{}是\df{可修改的},除非它类型为const限定或是一个函数类型。

\begin{note}
  尝试通过一个不可修改\lvalue{}或通过一个\rvalue{}来修改一个对象的程序是
  \illform{}的。
\end{note}

\paragraph{} % 11
如果程序尝试通过类型不与以下类型之一相似的\glvalue{}访问(\ref{defns.access})。
一个对象所存值,则行为未定义:\footnote{该列表旨在指明对象可以或不可以有别名的情
形。}
\begin{enumerate}
  \item 该对象的动态类型,
  \item 对应于对象动态类型的有符号或无符号类型,或者
  \item 一个\tm{char},\tm{unsigned char}或\tm{std::byte}类型。
\end{enumerate}
如果程序使用在其生命期内不表示\nt{cv} \tm{U}类型对象的\glvalue{}实参来调用类型
\tm{U}的联合的缺省拷贝/移动构造函数或者拷贝/移动赋值运算符,则行为未定义。

\begin{note}
  与\c{}不同,\cpp{}无类类型的访问。
\end{note}
