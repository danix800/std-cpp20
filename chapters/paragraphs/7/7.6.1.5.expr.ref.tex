% 7.6.1.5 Class member access [expr.ref]
\paragraph{} % 1
后缀表达式后跟点\tm{.}或箭头\tm{->},可选地跟上关键字\tm{template}
(\ref{temp.names}),随后再跟上一个\nt{id-expression},是一个后缀表达式。点或
箭头前的后缀表达式被求值;\footnote{如果类成员访问表达式被求值,子表达式也将被
求值,即使结果对于确定完整后缀表达式不是必须的,比如如果\nt{id-expression}表示
静态成员。}求值的结果,连同\nt{id-expression},确定整个后缀表达式的结果。

\paragraph{} % 2
对于第一个选项(点),第一表达式应该是广义左值。对于第二个选项(箭头),第一表
达式应该是一个指针类型的纯右值。表达式\tm{E1->E2}转换为等价形式
\tm{(*(E1)).E2};\ref{expr.ref}的剩余部分只描述第一个选项(点)。\footnote{注意
\tm{(*(E1))}是一个左值。}

\paragraph{} % 3
\nt{postfix-expression}\tm{.}\nt{id-expression}为\tm{E1.E2},\tm{E1}称为
\nt{对象表达式}。如果对象表达式为标量类型,\tm{E2}应该确定同类型伪析构(忽略cv
限定),并且\tm{E1.E2}是一个左值,类型为“返回\tm{void}的函数()”。

\begin{note}
  该值只能用于函数调用(\ref{expr.prim.id.dtor})。
\end{note}

\paragraph{} % 4
否则,对象表达式应该具有类类型。该类类型应该是完整类型,除非类成员访问出现在类
的定义中。

\begin{note}
  如果类不完整,完整类类型中的查询应该引用的是相同声明
  (\ref{basic.scope.class})。
\end{note}

\nt{id-expression}应该确定的是一个类或其基类的成员。

\begin{note}
  因在类作用域(第\ref{class}条)中插入类名,类名也当作是该类的嵌套成员。
\end{note}

\begin{note}
  \ref{basic.lookup.classref}描述了\tm{.}和\tm{->}运算符之后的名字如何查询。
\end{note}

\paragraph{} % 5
如果\tm{E2}是一个位域,则\tm{E1.E2}也是一个位域。\tm{E1.E2}的类型和值范畴按如下
确定。在\ref{expr.ref}剩余部分,\nt{cq}表示有\tm{const}或不存在\tm{const},
\nt{vq}表示存在\tm{volatile}或不存在\tm{volatile}。\nt{cv}表示\nt{6.8.4}中
定义的任意cv限定符集合。

\paragraph{} % 6
如果\tm{E2}定义为具有类型“\tm{T}的引用”,则\tm{E1.E2}是一个左值;\tm{E1.E2}的类
型为\tm{T}。否则适用以下规则之一。
\begin{enumerate}
  \item 如果\tm{E2}是一个静态数据成员且\tm{E2}的类型为\tm{T},则\tm{E1.E2}是一
        个左值;表达式表示类的命名成员。\tm{E1.E2}的类型为\tm{T}。
  \item 如果\tm{E2}是一个非静态数据成员且\tm{E1}的类型为``\nt{cq1 vq1} \tm{X}''
        并且\tm{E2}的类型为``\nt{cq2 vq2} \tm{T}'',则表达式表示的是第一个表达
        式所表示的对象的对应成员子对象。如果\tm{E1}是个左值,则\tm{E1.E2}是个左
        值;否则\tm{E1.E2}是个\xvalue。设符号\nt{vq12}表示\nt{vq1}和\nt{vq2}的
        ``并集'';即如果\nt{vq1}或\nt{vq2}是\tm{volatile},则\nt{vq12}为
        \tm{volatile}。类似的,设符号\nt{cq12}表示\nt{cq1}和\nt{cq2}的``并集'';
        即如果\nt{cq1}或\nt{cq2}为\tm{const},则\nt{cq12}为\tm{const}。如果
        \tm{E2}声明为\tm{mutable}成员,则\tm{E1.E2}的类型为
        ``\nt{vq12} \tm{T}''。如果\tm{E2}未声明为\tm{mutable}成员,则\tm{E1.E2}
        的类型为``\nt{cq12} \nt{vq12} \tm{T}''。
  \item 如果\tm{E2}是一个(可能重载的)成员函数,则使用函数重载解析
        (\ref{over.match})来选择\tm{E2}所引用的函数。\tm{E1.E2}的类型即
        \tm{E2}的类型,且\tm{E1.E2}引用的是\tm{E2}所引用的函数。
        \begin{enumerate}
          \item 如果\tm{E2}引用静态成员函数,则\tm{E1.E2}是左值。
          \item 否则(当\tm{E2}引用一个非静态成员函数时),\tm{E1.E2}是一个
                \prvalue。该表达式仅可用于一个成员函数调用(\ref{class.mfct})
                的左操作数。

                \begin{note}
                  忽略包含于此表达式之外的冗余括号集(\ref{expr.prim.paren})。
                \end{note}
        \end{enumerate}
  \item 如果\tm{E2}是一个嵌套类型,则\tm{E1.E2}为非法。
  \item 如果\tm{E2}是一个成员枚举,且\tm{E2}的类型为\tm{T},则表达式\tm{E1.E2}
        是一个\prvalue。\tm{E1.E2}的类型为\tm{T}。
\end{enumerate}

\paragraph{} % 7
如果\tm{E2}是一个非静态数据成员或一个非静态成员函数,且如果\tm{E2}作为其直接成
员的类是\tm{E2}所确定类(\ref{class.access.base})的有歧义基类
(\ref{class.member.lookup}),则程序为非法。

\begin{note}
  如果所确定类是对象表达式的类类型的有歧义基类,则程序也是非法;
  见\ref{class.access.base}。
\end{note}
