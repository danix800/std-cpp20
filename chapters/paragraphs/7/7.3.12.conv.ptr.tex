% 7.3.12 Pointer conversions [conv.ptr]
\paragraph{} % 1
一个\df{\nullp{}常量}是一个整型字面值(\ref{lex.icon}),其值为零或
\tm{std::nullptr\_t}类型的一个\prvalue{}。一个\nullp{}常量可以转换为一个指针类
型;结果为该类型的\nullp{}值(\ref{basic.compound})且不同于任何其他对象指针值或
函数指针值。这样的转换称为\df{\nullp{}转换}。同一个类型的两个\nullp{}值比较应该
相等。一个\nullp{}常量到cv限定类型指针的转换是单个转换,而不是一个指针转换后加上
一个限定转换(\ref{conv.qual})。一个整型的\nullp{}常量可以转换为
\tm{std::nullptr\_t}类型的\prvalue{}。

\begin{note}
  所产生的\prvalue{}不是一个\nullp{}值。
\end{note}

\paragraph{} % 2
“\nt{cv} \tm{T}指针”类型的\prvalue{},其中\tm{T}是一个对象类型,可以转换为一个
“\nt{cv} \tm{void}指针”类型的\prvalue{}。指针值(\ref{basic.compound})在该转换
下不变。

\paragraph{} % 3
“\nt{cv} \tm{D}指针”类型的\prvalue{},其中\tm{D}是一个完整类类型,可以转换为
“\nt{cv} \tm{B}指针”类型的\prvalue{},其中\tm{B}是\tm{D}的一个基类
(\ref{class.derived})。如果\tm{B}是\tm{D}的一个不可访问(\ref{class.access})
或有歧义(\ref{class.member.lookup})基类,则需要此转换的程序为\illform{}。转换
结果是派生类对象的基类子对象指针。\nullp{}值可以转换为目标类型的\nullp{}值。
