% 7.5.4.2 Unqualified names [expr.prim.id.unqual]
\synsym{unqualified-id}
  \synprd{\nt{identifier}}
  \synprd{\nt{operator-function-id}}
  \synprd{\nt{conversion-function-id}}
  \synprd{\nt{literal-operator-id}}
  \synprd{\tm{\tat} \nt{type-name}}
  \synprd{\tm{\tat} \nt{decltype-specifier}}
  \synprd[]{\nt{template-id}}

\paragraph{} % 1
一个\nt{identifier}仅是一个\nt{id-expression},如果其经过适当地声明
(第\ref{dcl.dcl}条)或如果其作为一个\nt{declarator-id}的一部分出现。一个确定协
程形参的\nt{identifier}引用该形参的拷贝(\ref{dcl.fct.def.coroutine})。

\begin{note}
  对于\nt{operator-function-id},见\ref{over.oper};对于
  \nt{conversion-function-id},见\ref{class.conv.fct};对于
  \nt{literal-operator-id}(\ref{over.literal});对于\nt{template-id},见
  \ref{temp.names}。一个前缀为\tm{\tat}的\nt{type-name}或\nt{decltype-specifier}
  表示所确定类型的析构函数;见\ref{expr.prim.id.dtor}。在一个非静态成员函数定义
  中,一个确定非静态成员的\nt{identifier}被转换成一个类成员访问表达式
  (\ref{class.mfct.non-static})。
\end{note}

\paragraph{} % 2
结果为标识符所表示的实体。如果该实体是一个局部实体且从出现\nt{unqualified-id}的
声明区域中的未求值操作数外部对其进行引用,将导致某些中间的\nt{lambda-expression}
被拷贝抓取(\ref{expr.prim.lambda.capture}),则表达式的类型为确定非静态数据成员
的类成员访问表达式(\ref{expr.ref})的类型,该类成员将在最内层这种中间
\nt{lambda-expression}闭包对象中为进行此抓取而声明。

\begin{note}
  如果该\nt{lambda-expression}未声明为\tm{mutable},则这种标识符的类型通常为
  \tm{const}限定的。
\end{note}

表达式的类型为结果的类型。

\begin{note}
  如果实体是一个\tm{T}类型的模板形参(\ref{temp.param})的模板形参对象,则表达式
  的类型为\tm{const T}。
\end{note}

\begin{note}
  如果是cv限定或是一个引用类型则该类型将按\ref{expr.type}进行调整。
\end{note}

如果实体是一个函数,变量,结构化绑定(\ref{dcl.struct.bind}),数据成员或模板形
参对象则表达式是一个\lvalue{},否则为\prvalue{}(\ref{basic.lval});如果标识符
表示一个位域则它也是一个位域。

\begin{example}
  \begin{lstlisting}
  void f() {
    float x, &r = x;
      [=] {
      decltype(x) y1;         // y1 has type float
      decltype((x)) y2 = y1;  // y2 has type float const& because this lambda
                              // is not mutable and x is an lvalue

      decltype(r) r1 = y1;    // r1 has type float&
      decltype((r)) r2 = y2;  // r2 has type float const&
    };
  }
  \end{lstlisting}
\end{example}
