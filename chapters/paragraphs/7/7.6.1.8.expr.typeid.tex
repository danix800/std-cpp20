% 7.6.1.8 Type identification [expr.typeid]
\paragraph{} % 1
\tm{typeid}表达式的结果是静态类型\tm{const std::type_info} (17.7.3) 和
动态类型\tm{const std::type_info}或\tm{const} \nt{name}的\lvalue{},其中
\nt{name}是从\tm{std::type_info}公有派生的实现定义的类,保留\ref{type.info}中所
述行为。\footnote{该类的推荐名称是\tm{extend_type_info}。}\lvalue{}引用的对象的
生命周期延伸到程序结束。未指定是否在程序结束时为\tm{std::type_info}对象调用析构
函数。

\paragraph{} % 2
当\tm{typeid}应用于类型为多态类类型(\ref{class.virtual})的\glvalue{}时,结果
引用的是表示最终派生对象类型(\ref{intro.object})的\tm{std::type_info}对象(即
动态类型)\glvalue{}所指的。如果通过将一元\tm{*}运算符应用于指针\footnote{如果
\tm{p}是指针类型的表达式,那么\tm{*p}、\tm{(*p)}、\tm{*(p)}、\tm{((*p))}、
\tm{*((p))}等都满足这个要求。}来获得\glvalue{},并且该指针是空指针值
(\ref{basic.compound}),则\tm{typeid}表达式会抛出与\tm{std::bad_typeid}类型
(\ref{bad.typeid})的处理程序匹配的类型的异常(\ref{except.throw})。

\paragraph{} % 3
当\tm{typeid}应用于多态类类型的\glvalue{}以外的表达式时,结果引用的是表示表达式
静态类型的\tm{std::type_info}对象。\lvalue{}到\rvalue{}(\ref{conv.lval})、数
组到指针(\ref{conv.array})和函数到指针(\ref{conv.func})转换不适用于表达式。
如果表达式是\prvalue{},则应用临时具化转换(\ref{conv.rval})。该表达式是一个未
求值的操作数(\ref{expr.prop})。

\paragraph{} % 4
当\tm{typeid}应用于\nt{type-id}时,结果引用的是表示\nt{type-id}类型的
\tm{std::type_info}对象。如果\nt{type-id}的类型是对可能cv限定类型的引用,则
\tm{typeid}表达式的结果引用表示无cv限定引用类型的\tm{std::type_info}对象。如果
\nt{type-id}的类型是类类型或对类类型的引用,则该类应是完全定义的。

\begin{note}
  \nt{type-id}不能用\nt{cv-qualifier-seq}或\nt{ref-qualifier}表示函数类型
  (\ref{dcl.fct})。
\end{note}

\paragraph{} % 5
如果表达式或\nt{type-id}的类型是cv限定类型,则\tm{typeid}表达式的结果引用表示无
cv限定类型的\tm{std::type_info}对象。

\begin{example}
  \begin{lstlisting}
  class D { /* ... */ };
  D d1;
  const D d2;

  typeid(d1) == typeid(d2);       // yields true
  typeid(D)  == typeid(const D);  // yields true
  typeid(D)  == typeid(d2);       // yields true
  typeid(D)  == typeid(const D&); // yields true
  \end{lstlisting}
\end{example}

\paragraph{} % 6
如果在使用\tm{typeid}之前未导入或包含标头\tm{<typeinfo>}(\ref{type.info}),则
程序\illform{}。

\paragraph{} % 7
\begin{note}
  第\ref{class.cdtor}节描述了应用于正在构造或销毁的对象的\tm{typeid}的行为。
\end{note}
