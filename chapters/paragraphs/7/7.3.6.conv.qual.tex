% 7.3.6 Qualification conversions [conv.qual]
\paragraph{} % 1
一个类型\tm{T}的\df{cv分解}指一个\nt{cv\tsub{i}}和\nt{P\tsub{i}}的序列使得\tm{T}
为                                                                          \par
\mbox\qquad{“\nt{cv\tsub{0} P\tsub{0} cv\tsub{1} ... \nt{P\tsub{1}
             \nt{cv\tsub{n}}}} \tm{U}”对\nt{n $\ge$ 0},}                   \par
其中每一个\nt{cv\tsub{i}}是一个cv限定(\ref{basic.type.qualifier})的集合,且每
一个\nt{P\tsub{i}}是一个“指向...”(\ref{dcl.ptr}),“类型为...的类\nt{C\tsub{i}}
的成员指针”(\ref{dcl.mptr}),“\nt{N\tsub{i}}的数组”,或者“...的未知上界数组”
(\ref{dcl.array})。如果\nt{P\tsub{i}}表示一个数组,则元素类型上的cv限定符
\nt{cv\tsub{i+1}}也被当作数组的cv限定符\nt{cv\tsub{i}}。

\begin{example}
  由\nt{type-id} \tm{const int **}所表示的类型
\end{example}

第一个之后的cv限定符\nt{n}元组为\tm{T}的最长cv分解,即\nt{cv\tsub{1},cv\tsub{2}
,...,cv\tsub{n}}称为\tm{T}的\df{cv限定签名}。

\paragraph{} % 2
两个类型\tm{T1}和\tm{T2}是\df{相似的},如果它们具有的cv分解具有相同的\nt{n},使
得对应的\nt{P\tsub{i}}成分要么是相同的,要么其中一个是“\nt{N\tsub{i}}的数组”,另
一个是“未知上界的数组”,且\tm{U}所表示的类型相同。

\paragraph{} % 3
两个类型\tm{T1}和\tm{T2}的\df{cv组合类型}为与\tm{T1}相似的类型\tm{T2},使得其cv
分解:
\begin{enumerate}
  \item 对每一个$i > 0$,$cv^3_i$为$cv^1_i$和$cv^2_i$的并集,
  \item 如果$P^1_i$或$P^2_i$为“未知上界的数组”,$P^3_i$为“未知上界的数组”,否则
        其为$P^1_i$,且
  \item 如果所产生的$cv^3_i$不同于$cv^1_i$或$cv^2_i$,或者产生的$P^3_i$不同于
        $P^1_i$或$P^2_i$,则向对$0 < k < i$的每一个$cv^3_k$添加\tm{const}。
\end{enumerate}
其中$cv^j_i$和$P^j_i$为\tm{T}\nt{j}的cv分解的成分。一个类型\tm{T1}的\prvalue{}可
以转换为类型\tm{T2},如果\tm{T1}和\tm{T2}的cv组合类型为\tm{T2}。

\begin{note}
  如果程序能将类型\tm{T**}的指针赋给一个类型\tm{const T**}的指针(即如果允许以下
  的行\#1),则不恰当的修改一个常对象将成为可能(如\#2所做的那样)。比如,
  \begin{lstlisting}
  int main() {
    const char c = ’c’;
    char* pc;
    const char** pcc = &pc;   // #1: not allowed
    *pcc = &c;
    *pc = ’C’;                // #2: modifies a const object
  }
  \end{lstlisting}
\end{note}

\begin{note}
  给定相似类型\tm{T1}和\tm{T2},该结构确保二者均可转换为\tm{T1}和\tm{T2}的cv组合
  类型。
\end{note}

\paragraph{} % 4
\begin{note}
  类型“\nt{cv1} \tm{T}的指针”的\prvalue{}可以转换成类型“\nt{cv2} \tm{T}的指针”的
  \prvalue{},如果“\nt{cv2} \tm{T}”比“\nt{cv1} \tm{T}”具有更强cv限定。类型为
  “\nt{cv1} \tm{T}类型的\tm{T}的成员指针”的\prvalue{}可以转换为一个类型为
  “\nt{cv1} \tm{T}类型的\tm{T}的成员指针”,如果“\nt{cv2} \tm{T}”比
  “\nt{cv1} \tm{T}”具有更强cv限定。
\end{note}

\paragraph{} % 5
\begin{note}
  函数类型(含用于成员函数指针类型的那些)永远不会cv限定(\ref{dcl.fct})。
\end{note}
