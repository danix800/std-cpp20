% 7.2.2 Type [expr.type]
\paragraph{} % 1
如果一个表达式最初具有“\tm{T}的引用”类型(\ref{dcl.ref},\ref{dcl.init.ref}),
则在任何进一步分析前类型被调整为\tm{T}。表达式指明引用所表示的对象或函数,且依据
表达式,表达式是一个\lvalue{}或一个\xvalue{}。

\begin{note}
  在引用生命期开始前或结束后,行为未定义(见\ref{basic.life})。
\end{note}

\paragraph{} % 2
如果\prvalue{}最初具有“\nt{cv} \tm{T}”类型,其中\tm{T}是一个cv未限定非类,非数组
类型,则在进一步分析前表达式的类型调整为\tm{T}。

\paragraph{} % 3
两个类型分别为\tm{T1}和\tm{T2}(其中至少一个是指针或成员指针类型或
\tm{std::nullptr\_t})的操作数\tm{p1}和\tm{p2}的\df{复合指针类型}为:
\begin{enumerate}
  \item 如果\tm{p1}和\tm{p2}均为\nullp{}常量,则为\tm{std::nullptr\_t};
  \item 如果\tm{p1}或\tm{p2}之一是\nullp{}常量,则分别为\tm{T2}或\tm{T1};
  \item 如果\tm{T1}或\tm{T2}为“指向\nt{cv} \tm{void}的指针”而另一个类型为“指向
        \nt{cv2} \tm{T}的指针”,其中\tm{T}是一个对象类型或\tm{void},则为“指向
        \nt{cv12} \tm{void}的指针”,其中\nt{cv12}为\nt{cv1}和\nt{cv2}的并集;
  \item 如果\tm{T1}中\tm{T2}为“指向\tm{noexcept}函数的指针”且另一个类型为“函数指
        针”,此外其中的函数类型是相同的,则为“函数指针”;
  \item 如果\tm{T1}是“指向\nt{cv1} \tm{C1}的指针”,\tm{T2}是“指向\nt{cv2}
        \tm{C2}的指针”,其中\tm{C1}与\tm{C2}引用相关或\tm{C2}与\tm{C1}引用相关
        (\ref{dcl.init.ref}),则分别为\tm{T1}和\tm{T2}的cv组合类型
        (\ref{conv.qual})或\tm{T2}和\tm{T1}的cv组合类型;
  \item 如果\tm{T1}或\tm{T2}为“指向\tm{C}的函数类型成员的指针”,另一个类型为“指
        向\tm{C2}的\tm{noexcept}函数成员指针”,且\tm{C1}与\tm{C2}引用相关,或者
        \tm{C2}与\tm{C1}引用相关(\ref{dcl.init.ref}),此外其中的函数类型是相同
        的,则分别为“指向\tm{C2}的函数类型指针”或者“指向\tm{C1}的函数类型成员指
        针”;
  \item 如果对某些非函数类型\tm{U},\tm{T1}为“指向\tm{C1}的\nt{cv1} \tm{U}类型的
        指针”且\tm{T2}为“指向\tm{C2}的\nt{cv2} \tm{U}类型的成员指针”,其中
        \tm{C1}与\tm{C2}引用相关,或者\tm{C2}与\tm{C1}引用相关
        (\ref{dcl.init.ref}),则分别为\tm{T2}和\tm{T1}的cv组合类型
        (\ref{conv.qual})或\tm{Tl}和\tm{T2}的cv组合类型;
  \item 如果\tm{T1}和\tm{T2}为相似类型(\ref{conv.qual}),则为\tm{T1}和\tm{T2}
        的组合类型;
  \item 否则,需要确定一个复合指针类型的程序为\illform{}。
\end{enumerate}
\begin{example}
  \begin{lstlisting}
    typedef void *p;
    typedef const int *q;
    typedef int **pi;
    typedef const int **pci;
  \end{lstlisting}
  \tm{p}和\tm{q}的复合指针类型为“\tm{const void}指针”;\tm{pi}和\tm{pci}的复合指
  针类型为“\tm{const int}的\tm{const}指针的指针”。
\end{example}
