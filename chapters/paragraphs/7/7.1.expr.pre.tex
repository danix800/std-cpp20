% 7.1 Preamble [expr.pre]
\paragraph{} % 1
\begin{note}
  第\ref{expr}条定义表达式语法,求值顺序及语义。\footnote{运算符的优先级未直接指
  定,但可从语法中推导出来。}一个表达式是指定一个运算的运算符和操作数序列。一个
  表达式可以产生一个值,且可以产生副作用。
\end{note}

\paragraph{} % 2
\begin{note}
  运算符可以重载,即在应用于类类型(第\ref{class}条)或者枚举类型
  (\ref{dcl.enum})表达式时给予语义。重载运算符的使用按\ref{over.oper}中所述转
  换成函数调用。重载运算符遵循\ref{expr.compound}中指定的语法规则和求值顺序,但
  操作数类型和值范畴的要求被替换成函数调用的规则。运算符间的关系,比如\tm{++a}意
  味着\tm{a+=1},对重载运算符不保证成立(\ref{over.oper})。
\end{note}

\paragraph{} % 3
子条款\ref{expr.compound}定义了运算符应用于未重载类型时的效果。运算符重载不得修
改\df{内置运算符}的规则,即适用于本标准为其定义类型的运算符。但这些内置运算符会
参与重载解析,作为该过程的一部分,将在必要时考虑用户定义转换,将操作数转换为适用
于内置运算符的类型。如果选择了内置运算符,则在根据第\ref{expr.compound}节中的规
则进一步考虑该运算之前,将对这些操作数进行此类转换;见\ref{over.match.oper},
\ref{over.built}。

\paragraph{} % 4
如果在表达式求值过程中,结果在数学上无定义或不在其类型可表示值范围内,则行为未定
义。

\begin{note}
  除零,以零为除数来取余或所有其他符点异常的处理在机器间各有不同,且有时可由库函
  数来调整。
\end{note}

\paragraph{} % 5
\begin{note}
  只有在运算符确实具有结合性或可交换的情况下,实现才可以根据通常的数学规则对运算
  符进行重新组合。\footnote{重载的运算符永远不会认为是结合的或可交换的。}例如,
  在以下片段中
  \begin{codeenv}
    \code{int a, b;}
    \code{/* ... */}
    \code{a = a + 32760 + b + 5;}
  \end{codeenv}
  因为这些运算符的结合性和优先级,表达式语句的行为与
  \begin{codeenv}
    \code{a = (((a + 32760) + b) + 5);}
  \end{codeenv}
  完全一样。因此,和\tm{(a + 32760)}的结果加到\tm{b},该结果再加到\tm{5},所产生
  的值赋给\tm{a}。在溢出产生异常和\tm{int}可表示值范围为\tm{[-32768, +32767]}的
  机器上,实现不能将表达式重写为
  \begin{codeenv}
    \code{a = ((a + b) + 32765);}
  \end{codeenv}
  因为若\tm{a}和\tm{b}的值分别为-32754和-15,则和\tm{a + b}将产生异常,而原表达
  式则不会;表达式也不能写成
  \begin{codeenv}
    \code{a = ((a + 32765) + b);}
  \end{codeenv}
  或
  \begin{codeenv}
    \code{a = (a + (b + 32765));}
  \end{codeenv}
  因\tm{a}和\tm{b}的值可以分别为4和-8,或者-17和12。但在溢出不产生异常和溢出结果
  可逆的机器上,以上表达式语句可以被实现按以上任一种方式重写,因为会产生相同的结
  果。
\end{note}

\paragraph{} % 6
符点运操作数的值和符点表达式的结果可以在比所需类型更大精度和范围内表示;类型不会
因此而改变。\footnote{转换和赋值运算符仍必须按\ref{expr.type.conv},
\ref{expr.cast},\ref{expr.static.cast}和\ref{expr.ass}中所述进行其特定的转换。}
