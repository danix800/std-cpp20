% 7.5.2 This [expr.prim.this]
\paragraph{} % 1
关键字\tm{this}确定一个非静态成员函数(\ref{class.this})调用或非静态数据成员初
始化(\ref{class.mem})求值的对象指针。

\paragraph{} % 2
如果一个声明声明了一个类\tm{X}的一个成员函数或成员函数模板,则表达式\tm{this}是
一个介于可选的\nt{cv-qualifier-seq}和\nt{funtion-definition},
\nt{member-declarator}或\nt{declarator}之间的类型为“\nt{cv-qualifier-seq} \tm{X}
的指针”的\prvalue{}。它不应该出现在可选的\nt{cv-qualifier-seq}之前,且不应该出现
静态成员函数声明中(尽管其类型和值范畴与在非静态成员函数中一样有定义)。

\begin{note}
  这是因为声明匹配直到完整声明子已知时才会进行。
\end{note}

\begin{note}
  在一个\nt{trailing-return-type}中,不需要所声明的类是完整的来进行类成员访问
  (\ref{expr.ref})。后声明的类成员不可见。

  \begin{example}
    \begin{lstlisting}
    struct A {
    char g();
      template<class T> auto f(T t) -> decltype(t + g())
        { return t + g(); }
    };
    template auto A::f(int t) -> decltype(t + g());
    \end{lstlisting}
  \end{example}

\end{note}

\paragraph{} % 3
否则,如果一个\nt{member-declarator}声明了类\tm{X}的一个非静态数据成员
(\ref{class.mem}),则表达式\tm{this}是介于可选缺省成员初始化
(\ref{class.mem})之间的一个类型为“\tm{X}的指针”的\prvalue{}。在
\nt{member-declarator}中它不应该出现在其他位置。

\paragraph{} % 4
表达式\tm{this}不应该出现在任何其他上下文中。

\begin{example}
  \begin{lstlisting}
    class Outer {
      int a[sizeof(*this)];             // error: not inside a member function
      unsigned int sz = sizeof(*this);  // OK: in default member initializer

      void f() {
        int b[sizeof(*this)];           // OK

        struct Inner {
          int c[sizeof(*this)];         // error: not inside a member function of Inner
        };
      }
    };
  \end{lstlisting}
\end{example}
