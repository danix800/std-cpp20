% 7.5.7.4 Compound requirements [expr.prim.req.compound]
\synsym{compound-requirement}
  \synprd{\tm{\{} \nt{expression} \tm{\} noexcept}\nt{\tsub{opt}}
    \nt{return-type-requirement\tsub{opt}} \tm{;}}
\synsym{return-type-requirement}
  \synprd{\tm{->} \nt{type-constraint}}

\paragraph{} % 1
\nt{compound-requirement}断言\nt{expression} \tm{E}的合法性。模板实参(如有)替
换和语义属性验证按如下顺序进行:
\begin{enumerate}
  \item 进行\nt{expression}中的模板实参替换(如有)。
  \item 如果存在\tm{noexcept}说明符,\nt{E}不应该是潜在抛异常表达式
    (\ref{except.spec})。
  \item 如果存在\nt{return-type-requirement},则:
    \begin{enumerate}
      \item 进行\nt{return-type-requirement}中的模板实参替换(如有)。
      \item 应满足为\tm{decltype((E))}的\nt{type-constraint}直接声明的约束
        (\ref{temp.param})。

        \begin{example}
          给定概念\tm{C}和\tm{D},
          \begin{lstlisting}
  requires {
    { E1 } -> C;
    { E2 } -> D<A1, ..., An>;
  };
          \end{lstlisting}
          等价于
          \begin{lstlisting}
  requires {
    E1; requires C<decltype((E1))>;
    E2; requires D<decltype((E2)), A1, ..., An>; 
  };
          \end{lstlisting}
          (包括\nt{n}为零的情形)。
        \end{example}
    \end{enumerate}
\end{enumerate}

\paragraph{} % 2
\begin{example}
  \begin{lstlisting}
  template<typename T> concept C1 = requires(T x) {
    {x++};
  };
  \end{lstlisting}
  \tm{C1}中的\nt{compound-requirement}要求\tm{x++}是有效表达式。等价于
  \nt{simple-requirement} \tm{x++;}。
  \begin{lstlisting}
  template<typename T> concept C2 = requires(T x) {
    {*x} -> std::same_as<typename T::inner>;
  };
  \end{lstlisting}
  \tm{C2}中的\nt{compound-requirement}要求\tm{*x}为有效表达式,
  \tm{typename T::inner}为有效类型,且满足
  \tm{std::same_as<decltype((*x)), typename T::inner>}。
  \begin{lstlisting}
  template<typename T> concept C3 =
    requires(T x) {
      {g(x)} noexcept;
    };
  \end{lstlisting}
  \tm{C3}中的\nt{compound-requirement}要求\tm{g(x)}为有效表达式且\tm{g(x)}不抛
  异常。
\end{example}
