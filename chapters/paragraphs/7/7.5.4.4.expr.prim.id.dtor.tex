% 7.5.4.4 Destruction [expr.prim.id.dtor]
\paragraph{} % 1
如果\tm{T}是类(\ref{class.dtor}),则表示\tm{T}的析构函数的\nt{id-expression}
确定\tm{T}的析构函数,否则\nt{id-expression}确定的是一个
\nt{pseudo-destructor}。

\paragraph{} % 2
如果\nt{id-expression}确定一个伪析构函数,\tm{T}应该是一个标量类型,且
\nt{id-expression}应该以形成函数调用(\ref{expr.call})的
\nt{postfix-expression}的类成员访问(\ref{expr.ref})的右操作数的形式出现。

\begin{note}
  这样的调用结束对象的生命期(\ref{expr.call},\ref{basic.life})。
\end{note}

\newpage

\paragraph{} % 3
\begin{example}
  \begin{lstlisting}
  struct C { };
  void f() {
    C * pc = new C;
    using C2 = C;
    pc->C::~C2();   // OK, destroys *pc
    C().C::~C();    // undefined behavior: temporary of type C destroyed twice
    using T = int;
    0 .T::~T();     // OK, no effect
    0.T::~T();      // error: 0.T is a user-defined-floating-point-literal (5.13.8)
  }
  \end{lstlisting}
\end{example}
