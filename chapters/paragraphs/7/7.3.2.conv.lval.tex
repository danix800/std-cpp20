% 7.3.2 Lvalue-to-rvalue conversion [conv.lval]
\paragraph{} % 1
一个非函数,非数组类型\tm{T}的\glvalue{}(\ref{basic.lval})可以转换为一个
\prvalue{}。\footnote{由于历史原因,该转换被称作是“\lvalue{}到\rvalue{}”转换,即
使这些名字并不准确反应\ref{basic.lval}中所描述的术语。}如果\tm{T}是一个不完整类
型,则需要该转换的程序为\illform{}。如果\tm{T}是一个非类类型,则\prvalue{}的类型
是\tm{T}的cv非限定版本。否则,该\prvalue{}的类型为\tm{T}。\footnote{在\cpp{}类和
数组\prvalue{}中可以有cv限定类型。这与\isoc{}不同,在\isoc{}中非\lvalue{}永远不
会具有cv限定类型。}

\paragraph{} % 2
当一个\lvalue{}到\rvalue{}转换应用于一个表达式\nt{E}时,且要么
\begin{enumerate}
  \item \nt{E}不会潜在求值,或者
  \item \nt{E}的求值导致\nt{E}的潜在结果集成员\nt{E}\tsub{\tm{x}}的求值,且
        \nt{E}\tsub{\tm{x}}确定未被\nt{E}\tsub{\tm{x}}\odruse{}的变量\tm{x}
        (\ref{basic.def.odr}),
\end{enumerate}
所引用对象所包含的值不会被访问。

\begin{example}
  \begin{lstlisting}
    struct S { int n; };
    auto f() {
      S x { 1 };
      constexpr S y { 2 };
      return [&](bool b) { return (b ? y : x).n; };
    }
    auto g = f();
    int m = g(false);     // undefined behavior: access of x.n outside its lifetime
    int n = g(true);      // OK, does not access y.n
  \end{lstlisting}
\end{example}

\paragraph{} % 3
转换的结果根据以下规则确定:
\begin{enumerate}
  \item 如果\tm{T}是\nt{cv} \tm{std::nullptr\_t},则结果为\nullp{}常量
        (\ref{conv.ptr})。

        \begin{note}
          因转换不访问\glvalue{}所引用对象,即使\tm{T}为易失限定
          (\ref{intro.execution})也不会有副作用,且\glvalue{}可以引用一个联合
          的非活动成员(\ref{class.union})。
        \end{note}
  \item 否则,如果\tm{T}具有类类型,则转换拷贝初始化来自\glvalue{}的结果对象。
  \item 否则,如果该\glvalue{}所引用的对象包含一个无效指针值
        (\ref{basic.stc.dynamic.deallocation},\ref{basic.stc.dynamic.safety})
        ,则行为由实现定义。
  \item 否则,读取(\ref{defns.access})该\glvalue{}所指代的对象,且包含在对象中
        的值是该\prvalue{}结果。
\end{enumerate}

\paragraph{} % 4
\begin{note}
  见\ref{basic.lval}。
\end{note}
