% 7.5.5.2 Closure types [expr.prim.lambda.closure]
\paragraph{} % 1
\nt{lambda-expression}的类型(同时也是该闭包对象的类型)是一个唯一的无名非联合
类类型,称为\nt{闭包类型},其属性如下所述。

\paragraph{} % 2
闭包类型在包含对应的\nt{lambda-expression}的最小块作用域、类作用域或命名空间作
用域声明。

\begin{note}
  这确定了关联于该闭包类型的命名空间和类的集合(\ref{basic.lookup.argdep})。
  \nt{lambda-declarator}的形参类型不影响关联的命名空间和类。
\end{note}

闭包类型不是一个聚合类型(\ref{dcl.init.aggr})。实现可以定义与以下所述不一样的
闭包类型,只要这不会改变除以下所列的程序可观察行为:
\begin{enumerate}
  \item 闭包类型的大小和/或对齐,
  \item 闭包类型是否平凡可拷贝(\ref{class.prop}),或
  \item 闭包类型是否为一个标准布局类(\ref{class.prop})。
\end{enumerate}

实现不应向闭包类型添加右值引用类型的成员。

\paragraph{} % 3
一个\nt{lambda-expression}的闭包类型具有一个公有内联函数调用算子
(对于非泛型$\lambda$)或函数调用算子模板(对于泛型模板)(\ref{over.call}),
其形参及返回类型由\nt{lambda-expression}的\nt{parameter-declaration-clause}和
\nt{trailing-return-type}对应所述,其\nt{template-parameter-list}包含所指定的
\nt{template-parameter-list}(如果有)。函数调用算子模板的
\nt{requires-clause}为紧跟在\tm{<} \nt{template-parameter-list} \tm{>}之后的
\nt{requires-clause}(如果有)。函数调用算子或算子模板结尾的
\nt{requires-clause}为\nt{lambda-declarator}的\nt{requires-clause}(如果有)。

\begin{note}
  泛型$\lambda$的函数调用算子模板可以为简写函数模板(\ref{dcl.fct})。
\end{note}

\begin{example}
  \begin{lstlisting}
  auto glambda = [](auto a, auto&& b) { return a < b; };
  bool b = glambda(3, 3.14);          // OK

  auto vglambda = [](auto printer) {
    return [=](auto&& ... ts) {       // OK: ts is a function parameter pack
      printer(std::forward<decltype(ts)>(ts)...);

      return [=]() {
        printer(ts ...);
      };
    };
  };
  auto p = vglambda( [](auto v1, auto v2, auto v3)
                     { std::cout << v1 << v2 << v3; } );
  auto q = p(1, 'a', 3.14);           // OK: outputs 1a3.14
  q();                                // OK: outputs 1a3.14
  \end{lstlisting}
\end{example}

\paragraph{} % 4
当且仅当\nt{lambda-expression}的\nt{parameter-declaration-clause}后没有跟上
\tm{mutable},函数调用算子或算子模板被声明为\tm{const}
(\ref{class.mfct.non-static})。其即不是虚函数也不声明为\tm{volatile}。
\nt{lambda-expression}上指定的任何\nt{noexcept-specifier}应用于对应的的函数调用
算子或算子模板。\nt{lambda-declarator}中的\nt{attribute-specifier-seq}作用于对
应函数调用算子或算子模板的类型。如果对应的\nt{lambda-expression}的
\nt{parameter-declaration-clause}后跟上\tm{constexpr}或\nt{consteval},或者满足
常表达式函数(\ref{dcl.constexpr})的要求,则函数调用算子或任何给定算子模板特例
化是一个常表达式函数。如果对应\nt{lambda-expression}的
\nt{parameter-declaration-clause}后跟\tm{consteval}则它是一个直接函数
(\ref{dcl.constexpr})。

\begin{note}
  \nt{lambda-declarator}中引用的名字在\nt{lambda-expression}所在的上下文中进行
  查询。
\end{note}

\begin{example}
  \begin{lstlisting}
  auto ID = [](auto a) { return a; };
  static_assert(ID(3) == 3);                  // OK

  struct NonLiteral {
    NonLiteral(int n) : n(n) { }
    int n;
  };
  static_assert(ID(NonLiteral{3}.n == 3);     // error
  \end{lstlisting}
\end{example}

\paragraph{} % 5
\begin{example}
  \begin{lstlisting}
  auto monoid = [](auto v) { return [=] { return v; }; };
  auto add = [](auto m1) constexpr {
    auto ret = m1();
    return [=](auto m2) mutable {
      auto m1val = m1();
      auto plus = [=](auto m2val) mutable constexpr
                     { return m1val += m2val; };
      ret = plus(m2());
      return monoid(ret);
    };
  };
  constexpr auto zero = monoid(0);
  constexpr auto one = monoid(1);
  static_assert(add(one)(zero)() == one());   // OK

  // Since two below is not declared constexpr, an evaluation of its constexpr
  // member function call operator cannot perform an lvalue-to-rvalue conversion
  // on one of its subobjects (that represents its capture) in a constant
  // expression.
  auto two = monoid(2);
  assert(two() == 2);                         // OK, not a constant expression.
  static_assert(add(one)(one)() == two());    // error: two() is not a constant expression
  static_assert(add(one)(one)() == monoid()); // OK
  \end{lstlisting}
\end{example}

\paragraph{} % 6
\begin{note}
  函数调用算子或算子模板可由\nt{type-constraint}(\ref{temp.param})、
  \nt{requires-clause}(\ref{temp.pre})或一个结尾的\nt{requires-clause}
  (\ref{dcl.decl})进行约束(\ref{temp.constr.decl})。

  \begin{example}
    \begin{lstlisting}
    template <typename T> concept C1 = /* ... */;
    template <std::size_t N> concept C2 = /* ... */;
    template <typename A, typename B> concept C3 = /* ... */;

    auto f = []<typename T1, C1 T2> requires C2<sizeof(T1) + sizeof(T2)>
             (T1 a1, T1 b1, T2 a2, auto a3, auto a4) requires C3<decltype(a4), T2> {
      // T2 is constrained by a type-constraint.
      // T1 and T2 are constrained by a requires-clause, and
      // T2 and the type of a4 are constrained by a trailing requires-clause.
    };
    \end{lstlisting}
  \end{example}

\end{note}

\paragraph{} % 7
无\nt{lambda-clause},满足约束(如果有)的非泛型\nt{lambda-expression}的闭包类
型具有一个到函数指针的转换函数,所指函数具有\cpp{}语言链接(\ref{dcl.link}),
具有与闭包类型的函数调用算子相同的形参和返回类型。如果函数调用算子具有不抛异常
说明则转换为到“\tm{noexcept}函数指针”。该转换函数返回的值为一个函数\tm{F},调用
时,与在该闭包类型的缺省构造实例上调用该闭包类型的函数调用算子具有相同效果。如
果函数调用算子是一个常表达式函数,则\tm{F}是一个常表达式函数,如果函数调用算子
是一个直接函数则该函数是一个直接函数。

\paragraph{} % 8
对于无\nt{lambda-capture}的泛型$\lambda$,闭包类型具有一个到函数指针的转换函数
模板。该转换函数模板具有相同的模板形参列表,且函数指针具有与函数调用算子模板相
同的形参类型。函数指针的返回类型应与表示函数调用算子模板特例化返回类型的
\nt{decltype-specifier}行为一致。

\paragraph{} % 9
\begin{note}
  如果泛型表达式无\nt{trailing-return-type}或\nt{trailing-return-type}包含一个
  占位符类型,则必需进行对应函数调用算子模板的返回类型推导。对应特例化即使用与
  转换函数模板推导相同的模板实参进行的函数调用算子模板实例化。考虑以下代码:
  \begin{lstlisting}
    auto glambda = [](auto a) { return a; };
    int (*fp)(int) = glambda;
  \end{lstlisting}
  上述的\tm{glambda}的转换函数行为如同以下转换函数:
  \begin{lstlisting}
    struct Closure {
      template<class T> auto operator()(T t) const { /* ... */ }
      template<class T> static auto lambda_call_operator_invoker(T a) {
        // forwards execution to operator()(a) and therefore has
        // the same return type deduced
        /* ... */
      }
      template<class T> using fptr_t =
         decltype(lambda_call_operator_invoker(declval<T>())) (*)(T);

      template<class T> operator fptr_t<T>() const
        { return &lambda_call_operator_invoker; }
    };
  \end{lstlisting}
\end{note}

\begin{example}
  \begin{lstlisting}
  void f1(int (*)(int))  { }
  void f2(char (*)(int)) { }

  void g(int (*)(int))   { }    // #1
  void g(char (*)(char)) { }    // #2

  void h(int (*)(int))   { }    // #3
  void h(char (*)(char)) { }    // #4

  auto glambda = [](auto a) { return a; };
  f1(glambda);                  // OK
  f2(glambda);                  // error: ID is not convertible
  g(glambda);                   // error: ambiguous
  h(glambda);                   // OK: calls #3 since it is convertible from ID
  int& (*fpi)(int*) = [](auto* a) -> auto& { return *a; };  // OK
  \end{lstlisting}
\end{example}

\paragraph{} % 10
该转换函数模板的任何给定特例化的返回值为函数\tm{F},在调用后,与在该闭包类型的
缺省构造实例上调用该泛型$\lambda$的对应函数调用算子模板特例化具有相同效果。如果
对应特例化是一个常表达式函数,则\tm{F}是一个常表达式函数,如果对应函数调用算子
模板特例化是一个直接函数则\tm{F}是一个直接函数。

\begin{note}
  这会产生泛型$\lambda$体的隐式实例化。实例化的泛型$\lambda$返回类型和形参类型
  必须与函数指针的返回类型和形参类型相匹配。
\end{note}

\begin{example}
  \begin{lstlisting}
  auto GL = [](auto a) { std::cout << a; return a; };
  int (*GL_int)(int) = GL;        // OK: through conversion function template
  GL_int(3);                      // OK: same as GL(3)
  \end{lstlisting}
\end{example}

\paragraph{} % 11
转换函数或转换函数模板为公有、常表达式、非虚、非显式、const且具有不抛异常说明
(\ref{except.spec})。

\begin{example}
  \begin{lstlisting}
  auto Fwd = [](int (*fp)(int), auto a) { return fp(a); };
  auto C = [](auto a) { return a; };

  static_assert(Fwd(C,3) == 3);   // OK
  // No specialization of the function call operator template can be constexpr
  // (due to the local static).
  auto NC = [](auto a) { static int s; return a; };
  static_assert(Fwd(NC,3) == 3);  // error
  \end{lstlisting}
\end{example}

\paragraph{} % 12
\nt{lambda-expression}的\nt{compound-statement}产生函数调用算子的
\nt{function-body}(\ref{dcl.fct.def}),但为了名字查询(\ref{basic.lookup}),
确定\tm{this}(\ref{class.this})的类型和值并转换引用了非静态类成员的
\nt{id-expression}为使用\tm{(*this)}(\ref{class.mfct.non-static})的类成员访问
表达式,该\nt{compound-statement}在\nt{lambda-expression}的上下文中考虑。

\begin{example}
  \begin{lstlisting}
  struct S1 {
    int x, y;
    int operator()(int);
    void f() {
      [=]()->int {
        return operator()(this->x + y); // equivalent to S1::operator()(this->x + (*this).y)
                                        // this has type S1*
      };
    }
  };
  \end{lstlisting}
\end{example}

更进一步,变量\tm{__func__}在\nt{lambda-expression}的\nt{compound-statement}的
开始处隐式定义,其语义如\ref{dcl.fct.def.general}所述。

\paragraph{} % 13
关联于\nt{lambda-expression}的闭包类型,如果\nt{lambda-expression}具有一个
\nt{lambda-capture}则无缺省构造函数,否则具有一个缺省构造函数。其具有一个缺省拷
贝构造函数和一个缺省移动构造函数(\ref{class.copy.ctor})。如果
\nt{lambda-expression}具有一个\nt{lambda-capture},则其具有一个删除的拷贝赋值算
子,否则具有缺省拷贝和移动赋值算子(\ref{class.copy.assign})。

\begin{note}
  这些特殊成员函数正常地隐式定义,也会导致其被删除。
\end{note}

\paragraph{} % 14
关联于\nt{lambda-expression}的闭包类型具有一个隐式声明的析构函数
(\ref{class.dtor})。

\paragraph{} % 15
闭包类型的成员不应该显式实例化(\ref{temp.explicit}),显式特例化
(\ref{temp.expl.spec})或用于友元声明(\ref{class.friend})中。
