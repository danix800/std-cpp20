% 7.3.7 Integral promotions [conv.prom]
\paragraph{} % 1
整型转换阶(\ref{conv.rank})小于\tm{int}的除\tm{bool},\tm{char16\_t},
\tm{char32\_t}或\tm{wchar\_t}外的整型\prvalue{}可以转换为\tm{int}类型的
\prvalue{},如果\tm{int}可以表示源类型的所有值;否则,源\prvalue{}可以转换为
\tm{unsigned int}类型的\prvalue{}。

\paragraph{} % 2
类型为\tm{char16\_t},\tm{char32\_t}或\tm{wchar\_t}(\ref{basic.fundamental})的
\prvalue{}可以转换为可以表示其底层类型所有值的以下类型中第一个的\prvalue{}:
\tm{int},\tm{unsigned int},\tm{long int},\tm{unsigned long int},
\tm{long long int},或\tm{unsigned long long int}。如果该列表中类型都不能表示其
底层类型的所有值,一个\tm{char16\_t},\tm{char32\_t}或\tm{wchar\_t}类型的
\prvalue{}可以转换为其底层类型的\prvalue{}。

\paragraph{} % 3
一个底层类型不固定的无作用域枚举类型\prvalue{}可以转换为可表示其所有枚举值
(\ref{dcl.enum})的以下类型中的第一个:
\tm{int},\tm{unsigned int},\tm{long int},\tm{unsigned long int},
\tm{long long int},或\tm{unsigned long long int}。如是该列表中类型都不能表示枚
举的所有值,则一个无作用域枚举类型\prvalue{}可以转换为一个扩展整型的\prvalue{},
具有大于\tm{long long}的最低整型转换阶(\ref{conv.rank}),所有枚举的值可在其中
表示。如果存在两个这样的扩展类型,则取有符号的那个。

\paragraph{} % 4
底层类型固定(\ref{dcl.enum})的无作用域枚举类型\prvalue{}可以转换为可以表示其底
层类型的\prvalue{}。更进一步,如果整型提升可用于其底层类型,则一个底层类型固定的
无作用域枚举类型\prvalue{}也可以转换为提升后底层类型的\prvalue{}。

\paragraph{} % 5
一个整型位域(\ref{class.bit})\prvalue{}可以转换为\tm{int}类型的\prvalue{},如
果\tm{int}可以表示所有位域的值;否则,其可以转换为\tm{unsigned int},如果
\tm{unsigned int}可以表示位域的所有值。如果位域还是大,则不对其应用整型提升。如
果位域具有枚举类型,则为提升的目的将其当作该类型的任何其他值。

\paragraph{} % 6
一个\tm{bool}类型的\prvalue{}可以转换为一个\tm{int}类型的\prvalue{},其中
\tm{false}转换为零,\tm{true}转换为一。

\paragraph{} % 7
这些转换称为\df{整型提升}。
