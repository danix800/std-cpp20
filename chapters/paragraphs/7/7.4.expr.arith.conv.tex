% 7.4 Usual arithmetic conversions [expr.arith.conv]
\paragraph{} % 1
许多需要算术或枚举类型操作数的二元运算符会按类似方法引起转换并产生结果类型。其目
的是产生一个公共类型,也是结果的类型。该模式称为\df{常规算术转换},定义如下:
\begin{enumerate}
  \item 如果任一个操作数具有有作用域枚举类型(\ref{dcl.enum}),则不进行转换;如
        果另一个操作数不具有相同类型,则表达式为\illform{}。
  \item 如果任一个操作数具有\tm{long double}类型,则另一个应该转换为
        \tm{long double}。
  \item 否则,如果任一个操作数为\tm{double},则另一个应该转换为\tm{double}。
  \item 否则,如果任一个操作数为\tm{float},则另一个应该转换为\tm{float}。
  \item 否则,应对两个操作数进行整型提升(\ref{conv.prom})。\footnote{其结果就
        是,类型\tm{bool},\tm{char8\_t},\tm{char16\_t},\tm{char32\_t},
        \tm{wchar\_t}或枚举类型的操作数被转换为某个整型。}然后对提升后的操作数应
        用以下规则:
        \begin{enumerate}
          \item 如果两个操作数具有相同类型,则无须进一步转换。
          \item 否则,如果两个操作数都为有符号整型或都为无符号整型,则具有较低整
                型转换阶的操作数应该转换为具有较高转换阶的类型。
          \item 否则,如果具有无符号整型的操作数具有高于或相同于另一个操作数的转
                换阶则具有有符号整型的操作数应该转换为具有无符号操作数的类型。
          \item 否则,如果具有有符号整型操作数可以表示具有无符号整型操作数的全部
                值,则具有无符号整型操作数应该转换为有符号操作数的类型。
          \item 否则,两个操作数都应该转换为对应于有符号整型操作数类型的无符号整
                型。
        \end{enumerate}
\end{enumerate}

\paragraph{} % 2
如果其中一个操作数具有枚举类型,另一个操作数具有不同枚举类型,或具有符点类型,其
行为已被废弃(\ref{depr.arith.conv.enum})。
