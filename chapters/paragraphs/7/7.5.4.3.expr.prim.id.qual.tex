% 7.5.4.3 Qualified names [expr.prim.id.qual]
\synsym{qualified-id}
  \synprd{\nt{nested-name-specifier} \tm{template}\nt{\tsub{opt}
    unqualified-id}}
\synsym{nested-name-specifier}
  \synprd{\tm{::}}
  \synprd{\nt{type-name} \tm{::}}
  \synprd{\nt{decltype-specifier} \tm{::}}
  \synprd{\nt{nested-name-specifier identifier} \tm{::}}
  \synprd[]{\nt{nested-name-specifier} \tm{template}\nt{\tsub{opt}
    simple-template-id} \tm{::}}

\paragraph{} % 1
一个\nt{nested-name-specifier}中的\nt{decltype-specifier}所表示的类型应该是一个
类或枚举类型。

\paragraph{} % 2
表示一个类,后跟可选的关键字\tm{template}(\ref{temp.names})和该类
(\ref{class.mem})或其基类(\ref{class.derived})之一的成员名的
\nt{nested-name-specifier}是一个\nt{qualified-id};\ref{class.qual}描述了出现在
\nt{qualified-id}中的类成员名字查询。结果即该成员。结果类型为该成员的类型。如果
成员是一个静态成员函数或数据成员,则结果是一个左值,否则为纯右值。

\begin{note}
  类成员可在其潜在作用域(\ref{basic.scope.class})中的任意点通过一个
  \nt{qualified-id}进行引用。
\end{note}
\nt{type-name}\tm{::\tat}\nt{type-name}中的两个\nt{type-name}应引用同一个类型(
忽略cv限定);表示以此命名的类型的析构函数(\ref{expr.prim.id.dtor})。
\nt{qualified-id}中的\nt{unqualified-id}不应该是
\tm{\tat}\nt{decltype-specifier}。

\paragraph{} % 3
\nt{nested-name-specifier} \tm{::}表示全局命名空间。表示命名空间
(\ref{basic.namespace})的\nt{nested-name-specifier},后跟可选的关键字
\tm{template}(\ref{temp.names})和该命名空间成员(或通过\nt{using-directive}使
其可见的命名空间成员名),是一个\nt{qualified-id};\ref{namespace.qual}中描述了
出现在\nt{qualified-id}中的命名空间名字的查询。其结果即该成员。结果类型为该成员
的类型。如果成员是一个函数、变量或结构化绑定(\ref{dcl.struct.bind}),则结果是
一个左值,否则为纯右值。

\paragraph{} % 4
表示枚举(\ref{dcl.enum})的\nt{nested-name-specifier},后跟该枚举成员名,是一
个引起该枚举常量的\nt{qualified-id}。其结果是该枚举常量,结果类型为枚举的类型。
结果是一个纯右值。

\paragraph{} % 5
在一个\nt{qualified-id}中,如果\nt{unqualified-id}是一个
\nt{conversion-function-id},其\nt{conversion-type-id}首先在由\nt{qualified-id}
的\nt{nested-name-specifier}所表示的类中进行查询,如果找到,则使用该名字。否则
在\nt{qualified-id}所出现的整个上下文中进行查询。在两种查询中,只考虑其名字表示
类型或特例化为类型的模板的名字。
