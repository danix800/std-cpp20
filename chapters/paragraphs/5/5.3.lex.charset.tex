% 5.3 Character sets [lex.charset]
\paragraph{} % 1
\df{基本源码字符集}由96个字符组成:空格,水平制表符,垂直制表符,换页符
(form feed)和新行,加上以下91个图形字符:\footnote{基本源码字符集成员的字形用
于标识ISO/IEC 10646子集中的字符,对应于ASCII字符集。但源文件字符到源码字符集的映
射(翻译阶段\ref{tpit1}中所述)由实现定义,因此实现必须对源文件中基本源码字符符
如何表示进行记录。}

\mbox{\qquad \tm{a b c d e f g h i j k l m n o p q r s t u v w x y z}}      \par
\mbox{\qquad \tm{A B C D E F G H I J K L M N O P Q R S T U V W X Y Z}}      \par
\mbox{\qquad \tm{0 1 2 3 4 5 6 7 8 9}}                                      \par
% _ { } [ ] # ( ) < > % : ; . ? * + - / ^ & | ~ ! = , \ " '
\mbox{\qquad \tm{\_ \{ \} [ ] \# ( ) \tl{} \tg{} \% :\ ; .\ ?\ * + - / \tac{} \&
  | \tac{} !\ = , \bs{} \dq{} \sq}}

\paragraph{} % 2
结构\nt{universal-character-name}提供了一种命名其他字符的方法。

\synsym{hex-quad}
  \synprd{\nt{hexadecimal-digit hexadecimal-digit hexadecimal-digit
    hexadecimal-digit}}
\synsym{universal-character-name}
  \synprd{\tm{\bs{}u} \nt{hex-quad}}
  \synprd{\tm{\bs{}U} \nt{hex-quad hex-quad}}
一个\nt{universal-character-name}代表ISO/IEC 10646中的字符(如有),其码点为
\nt{universal-character-name}中的\nt{hexadecimal-digit}序列所表示的十六进制数。
如果该数字不是一个码点或者是一个替代码点则程序为\illform{}。非字符码点和保留码点
考虑为代表不同于任何ISO/IEC 10646字符的不同字符。如果一个\nt{character-literal}
或\nt{string-literal}(均包括于\nt{user-defined-literal}内)的
\nt{c-char-sequence},\nt{s-char-sequence}或\nt{r-char-sequence}之外的
\nt{universal-character-name}对应于一个控制字符或者基本源码字符集中的字符,则程
序为\illform{}。\footnote{类似于\nt{r-char-sequence}(\ref{lex.string})中的
\nt{universal-character-name}的字符序列不构成\nt{universal-character-name}。}

\begin{note}
  ISO/IEC 10646码点为区间[0, 10FFFF](十六进制)中的整数。替代码点为区间
  [D800, DFFF](十六进制)中的值。控制字符为码点位于区间[0, 1F]或[7F, 9F](十六
  进制)中的字符。
\end{note}

\paragraph{} % 3
\df{基本执行字符集}和\df{基本执行宽字符集}均应包含基本源码字符集的所有成员,加上
表示预警,退格和回车的控制字符,以及\df{零字符}(对应\df{零宽字符}),其值为0。
对每一个基本执行字符集,成员的值应该非负且彼此不同。对基本源码和基本执行字符集
中,以上十进制数字列表中\tm{0}之后的每一个字符的值应该比前一个大一。\df{执行字符
集}和\df{执行宽字符集}对应的分别为基本执行字符集和基本执行宽字符集的超集。执行字
符集成员和额外集合成员的值为语言环境特定的。
