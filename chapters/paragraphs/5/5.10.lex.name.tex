% 5.10 Identifiers [lex.name]
\synsym{identifier}
  \synprd{\nt{identifier-nondigit}}
  \synprd{\nt{identifier identifier-nondigit}}
  \synprd{\nt{identifier digit}}
\synsym{identifier-nondigit}
  \synprd{\nt{nondigit}}
  \synprd{\nt{universal-character-name}}
\synsym[one of]{nondigit}
  \synprd{\tm{a b c d e f g h i j k l m}}
  \synprd{\tm{n o p q r s t u v w x y z}}
  \synprd{\tm{A B C D E F G H I J K L M}}
  \synprd{\tm{N O P Q R S T U V W X Y Z \_}}
\synsym[one of]{digit}
  \synprd[]{\tm{0 1 2 3 4 5 6 7 8 9}}

\paragraph{} % 1
标识符指任意长字母和数字序列。标识符中的每一个\nt{universal-character-name}应该
表示编码在ISO/IEC 10646中,表\ref{tab:lex.name.allowed}中所列范围内的字符。起始
元素不能是编码在表\ref{tab:lex.name.disallowed}中所列范围内的
\nt{universal-character-name}。大小写字符不同。所有字符均有意义。\footnote{在链
接器不能接受扩展字符的系统上,\nt{universal-character-name}的编码可能用来形成有
效外部字符。比如,某些未用字符或序列可能用来编码\nt{universal-character-name}中
的\tm{\bs u}。扩展字符可能产生较长的外部标识符,但是\cpp{}不对外部标识符有效字母
数作翻译限制。在\cpp{}中,大小写字母在所有标识符中都不同,包括外部标识符。}

\begin{table}[h!]
  \centering
  \caption{允许的字符范围 [tab:lex.name.allowed]}
  \begin{tabular}{|lllll|}
    \hline
    \tm{00A8}        & \tm{00AA}        & \tm{00AD}        & \tm{00AF}        &
      \tm{00B2-00B5}                                                          \\
    \tm{00B7-00BA}   & \tm{00BC-00BE}   & \tm{00C0-00D6}   & \tm{00D8-00F6}   &
      \tm{00F8-00FF}                                                          \\
    \tm{0100-167F}   & \tm{1681-180D}   & \tm{180F-1FFF}   &                  &
                                                                              \\
    \tm{200B-200D}   & \tm{202A-202E}   & \tm{203F-2040}   & \tm{2054}        &
      \tm{2060-206F}                                                          \\
    \tm{2070-218F}   & \tm{2460-24FF}   & \tm{2776-2793}   & \tm{2C00-2DFF}   &
      \tm{2E80-2FFF}                                                          \\
    \tm{3004-3007}   & \tm{3021-302F}   & \tm{3031-D7FF}   &                  &
                                                                              \\
    \tm{F900-FD3D}   & \tm{FD40-FDCF}   & \tm{FDF0-FE44}   & \tm{FE47-FFFD}   &
                                                                              \\
    \tm{10000-1FFFD} & \tm{20000-2FFFD} & \tm{30000-3FFFD} & \tm{40000-4FFFD} &
      \tm{50000-5FFFD}                                                        \\
    \tm{60000-6FFFD} & \tm{70000-7FFFD} & \tm{80000-8FFFD} & \tm{90000-9FFFD} &
      \tm{A0000-AFFFD}                                                        \\
    \tm{B0000-BFFFD} & \tm{C0000-CFFFD} & \tm{D0000-DFFFD} & \tm{E0000-EFFFD} &
                                                                              \\
    \hline
  \end{tabular}
  \label{tab:lex.name.allowed}
\end{table}

\begin{table}[h!]
  \centering
  \caption{初始不允许的字符序列(组合字符)[tab:lex.name.disallowed]}
  \begin{tabular}{|llll|}
    \hline
    \tm{0300-036F} & \tm{1DC0-1DFF} & \tm{20D0-20FF} & \tm{FE20-FE2F}         \\
    \hline
  \end{tabular}
  \label{tab:lex.name.disallowed}
\end{table}

\paragraph{} % 2
表\ref{tab:lex.name.special}中的标识符在某些上下文中具有特殊意义。当在语法中引用
时,这些标识符是显式使用的,而不是使用\nt{identifier}语法的产生式。除另有说明,
否则对于给定\nt{identifier}是否具有特殊含义的任何歧义都是将符号解释为常规的
\nt{identifier}。

\begin{table}[h!]
  \centering
  \caption{具特殊意义的标识符[tab:lex.name.special]}
  \begin{tabular}{|llll|}
    \hline
    \tm{final} & \tm{import} & \tm{module} & \tm{override}                    \\
    \hline
  \end{tabular}
  \label{tab:lex.name.special}
\end{table}

\paragraph{} % 3
此外,某些标识符保留给\cpp{}实现使用,不应再作他用;无需诊断。
\begin{enumerate}
  \item 保留含双下划线\tm{\_\_}或以下划线开头后跟上大写字母的标识符给实现使用。
  \item 以下划线开始的标识符保留给实现用作全局命名空间名字。
\end{enumerate}
