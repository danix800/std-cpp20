% 5.13.3 Character literals [lex.ccon]
\synsym{character-literal}
  \synprd{\nt{encoding-prefix\tsub{opt}}
          \tm{\sq} \nt{c-char-sequence} \tm{\sq}}
\synsym[one of]{encoding-prefix}
  \synprd{\tm{u8 u U L}}
\synsym{c-char-sequence}
  \synprd{\nt{c-char}}
  \synprd{\nt{c-char-sequence} \nt{c-char}}
\synsym{c-char}
  \synprd{除单引号\tm{\sq}、反斜杠\tm{\bs}或新行字符以外的基本源码字符集成员}
  \synprd{\nt{escape-sequence}}
  \synprd{\nt{universal-character-name}}
\synsym{escape-sequence}
  \synprd{\nt{simple-escape-sequence}}
  \synprd{\nt{octal-escape-sequence}}
  \synprd{\nt{hexadecimal-escape-sequence}}
\synsym[one of]{simple-escape-sequence}
  \synprd{\tm{\bs\sq\ \ \bs\dq\ \ \bs?\ \ \bs\bs}}
  \synprd{\tm{\bs a \ \bs b \ \bs f \ \bs n \ \bs r \ \bs t \ \bs v}}
\synsym{octal-escape-sequence}
  \synprd{\tm{\bs} \nt{octal-digit}}
  \synprd{\tm{\bs} \nt{octal-digit octal-digit}}
  \synprd{\tm{\bs} \nt{octal-digit octal-digit octal-digit}}
\synsym{hexadecimal-escape-sequence}
  \synprd{\tm{\bs x} \nt{hexadecimal-digit}}
  \synprd[]{\nt{hexadecimal-escape-sequence} \nt{hexadecimal-digit}}

\paragraph{} % 1
不以\tm{u8},\tm{u},\tm{U}或\tm{L}开头的\nt{character-literal}为\nt{普通字符字
面值}。含有单个执行字符集可表示\nt{c-char}的普通字符字面值具有\tm{char}类型,其
值等于执行字符集中该\nt{c-char}编码的数值。包含多于一个\nt{c-char}的字符字面值为
\nt{多字符字面值}。多字符字面值以及含有执行字符集不能表示的单个\nt{c-char}的普通
字符字面值为条件支持,具有\tm{int}类型,具有实现定义值。

\paragraph{} % 2
以\tm{u8}开头的\nt{character-literal},如\tm{u8'w'},其类型为\tm{char8\_t},即
\nt{UTF-8字符字面值}。一个UTF-8字面字符的值等于其在ISO 10646中的码点值,如果该码
点值可以使用单个UTF-8编码单元表示的话

\begin{note} % 1
  即如果码点值处于区间[0, 7F](十六进制)中。
\end{note}

如果该值不能使用单个UTF-8编码单元表示,则程序为\illform{}。包含多个\nt{c-char}的
UTF-8字符字面值是\illform{}的。

\paragraph{} % 3
以字母\tm{u}开头的\nt{character-literal},如\tm{u'x'},其类型为\tm{char16\_t},
即\nt{UTF-16字符字面值}。UTF-16字面字符的值等于其ISO 10646中的码点值,如果该码点
可以使用单个16位码点单元表示的话。

\begin{note} % 2
  即如果码点值处于区间[0, FFFF](十六进制)中。
\end{note}

如果该值在单个16位编码单元内不可表示,则程序为\illform{}。包含多个\nt{c-char}的
UTF-16字符字面值是\illform{}的。

\paragraph{} % 4
以字母\tm{U}开头的\nt{character-literal},如\tm{U'y'},其类型为\tm{char32\_t},
即\nt{UTF-32字符字面值}。包含单个\nt{c-char}的UTF-32字面字符的值等于其在ISO
10646中的码点值。包含多个\nt{c-char}的\tm{char32\_t}字符字面值是\illform{}的。

\paragraph{} % 5
\begin{sloppypar}
以字母\tm{L}开头的\nt{character-literal}为\nt{宽字符字面值},比如\tm{L'z'}。宽字
符字面值类型为\tm{wchar\_t}。\footnote{适用于字符不能用单个字节表示的字符集。}包
含单个\nt{c-char}的宽字符字面值的值等于执行宽字符集中\nt{c-char}编码的数值,除非
\nt{c-char}不能用执行宽字符集表示,在这种情况下,该值由实现定义。
\end{sloppypar}

\begin{note} % 3
  类型\tm{wchar\_t}可以表示执行宽字符集所有成员(见\ref{basic.fundamental})。
\end{note}

包含多个\nt{c-char}的宽字面字符的值由实现定义。

\paragraph{} % 6
某些非图形字符、单引号\tm{'}、双引号\tm{"}、问号\tm{?},\footnote{对问号使用转义
是为了兼容\isocppfor{}和\isoc{}。}和反斜杠\tm{\bs}可以根据表
\ref{tab:lex.ccon.esc}来表示。双引号\tm{"}和问号\tm{?}可以表示为其自身,也可以分
别表示为转义序列\tm{\bs"}和\tm{\bs ?}。但单引号\tm{'}和反斜杠\tm{\bs}应分别由转
义序列\tm{\bs'}和\tm{\bs\bs}表示。表\ref{tab:lex.ccon.esc}中没有列出反斜杠后跟字
符的转义序列是条件支持的,具有实现定义的语义。一个转义序列指定单个字符。

\begin{table}[!ht]
  \centering
  \caption{转义序列[tab:lex:ccon.esc]}
  \begin{tabular}{|lll|}
    \hline
    新行       & \tm{NL(LF)} & \tm{\bs n}                                     \\
    水平制表符 & \tm{HT}     & \tm{\bs t}                                     \\
    垂直制表符 & \tm{VT}     & \tm{\bs v}                                     \\
    退格       & \tm{BS}     & \tm{\bs b}                                     \\
    回车       & \tm{CR}     & \tm{\bs r}                                     \\
    换页符     & \tm{FF}     & \tm{\bs f}                                     \\
    预警符     & \tm{BEL}    & \tm{\bs a}                                     \\
    反斜杠     & \tm{\bs}    & \tm{\bs\bs}                                    \\
    问号       & \tm{?}      & \tm{\bs ?}                                     \\
    单引号     & \tm{'}      & \tm{\bs'}                                      \\
    双引号     & \tm{"}      & \tm{\bs"}                                      \\
    八进制数   & \nt{ooo}    & \tm{\bs}\nt{ooo}                               \\
    十六进制数 & \nt{hhh}    & \tm{\bs x}\nt{hhh}                             \\
    \hline
  \end{tabular}
  \label{tab:lex.ccon.esc}
\end{table}

\paragraph{} % 7
转义\tm{\bs}\nt{ooo}由反斜杠后跟一个、两个或三个八进制数字组成,用于指定所需字符
的值。转义\tm{\bs}\nt{hhh}由反斜杠后跟\tm{x},一个或多个十六进制数字组成,用于指
定所需字符的值。十六进制序列中数字的个数没有限制。八进制或十六进制数字的序列分别
以非八进制数字或非十六进制数字的第一个字符结束。如果\nt{character-literal}的值不
在\tm{char}(对无前缀\nt{character-literal})或\tm{wchar\_t}(对前缀为\tm{L}的
\nt{character-literal})的实现定义范围内,则其值由实现定义。

\begin{note} % 4
  如果前缀为\tm{u}、\tm{u8}或\tm{U}的\nt{character-literal}的值超出其类型所定义
  的范围,则程序为\illform{}。
\end{note}

\paragraph{} % 8
一个\nt{universal-character-name}翻译为适当的执行字符集中的指定字符的编码。如果
没有这样的编码,则\nt{universal-character-name}将翻译为实现定义的编码。

\begin{note} % 5
  在翻译阶段\ref{tpit1}中,当源码文本中遇到实际扩展字符时将引入
  \nt{universal-character-name}。因此,所有扩展字符都用
  \nt{universal-character-name}来描述。但是,只要能获得相同结果,实际的编译器实
  现就可以使用自己的本地字符集。
\end{note}
