% 5.13.4 Floating-point literals [lex.fcon]
\synsym{floating-point-literal}
  \synprd{\nt{decimal-floating-point-literal}}
  \synprd{\nt{hexadecimal-floating-point-literal}}
\synsym{decimal-floating-point-literal}
  \synprd{\nt{fractional-constant exponent-part\tsub{opt}
    floating-point-suffix\tsub{opt}}}
  \synprd{\nt{digit-sequence exponent-part floating-point-suffix\tsub{opt}}}
\synsym{hexadecimal-floating-point-literal}
  \synprd{\nt{hexadecimal-prefix hexadecimal-fractional-constant
    binary-exponent-part}} % cont
  \synprd{\tm{\ \ \ \ }\nt{floating-point-suffix\tsub{opt}}}
  \synprd{\nt{hexadecimal-prefix hexadecimal-digit-sequence
    binary-exponent-part}} % cont
  \synprd{\tm{\ \ \ \ }\nt{floating-point-suffix\tsub{opt}}}
\synsym{fractional-constant}
  \synprd{\nt{digit-sequence\tsub{opt}} \tm{.} \nt{digit-sequence}}
  \synprd{\nt{digit-sequence} \tm{.}}
\synsym{hexadecimal-fractional-constant}
  \synprd{\nt{hexadecimal-digit-sequence\tsub{opt}} \tm{.}
    \nt{hexadecimal-digit-sequence}}
  \synprd{\nt{hexadecimal-digit-sequence} \tm{.}}
\synsym{exponent-part}
  \synprd{\tm{e} \nt{sign\tsub{opt} digit-sequence}}
  \synprd{\tm{E} \nt{sign\tsub{opt} digit-sequence}}
\synsym[one of]{sign}
  \synprd{\tm{+ -}}
\synsym{digit-sequence}
  \synprd{\nt{digit}}
  \synprd{\nt{digit-sequence} \tm{'}\nt{\tsub{opt} digit}}
\synsym[one of]{floating-point-suffix}
  \synprd[]{\tm{f l F L}}

\paragraph{} % 1
按表\ref{tab:lex.fcon.type}中所指定,一个\nt{floating-point-literal}的类型由其
\nt{floating-point-suffix}确定。

\begin{table}[!ht]
  \centering
  \caption{\nt{floating-point-literal}的类型[tab:lex.fcon.type]}
  \begin{tabular}{|ll|}
    \hline
    \tb{\nt{floating-point-suffix}} & \tb{类型}                               \\
    \hline\hline
    无                              & \tm{double}                             \\
    \tm{f}或\tm{F}                  & \tm{float}                              \\
    \tm{l}或\tm{L}                  & \tm{long double}                        \\
    \hline
  \end{tabular}
  \label{tab:lex.fcon.type}
\end{table}

\paragraph{} % 2
一个\nt{floating-point-literal}的\nt{有效位}(\df{significand})为
\nt{fractional-constant}或一个\nt{decimal-floating-point-literal}的
\nt{digit-sequence}或\nt{hexadecimal-fractional-constant}或一个
\nt{hexadecimal-floating-point-literal}的\nt{hexadecimal-digit-sequence}。在有效
位中,\nt{digit}或\nt{hexadecimal-digit}序列以及可选的小数点解释为以\nt{N}为基的
实数\nt{s},其中\nt{N}对\nt{decimal-floating-point-literal}为10,对
\nt{hexadecimal-floating-point-literal}为16。

\begin{note}
  在确定值时忽略任何可选的分隔用单引号。
\end{note}

如果一个\nt{exponent-part}或\nt{binary-exponent-part}存在,则
\nt{floating-point-literal}的指数\nt{e}为解释可选\nt{sign}和\nt{digit}序列为以
10为基的整数的结果。否则指数\nt{e}为0。对一个\nt{decimal-floating-point-literal}
字面值的缩放值为$s \times 10^e$,对一个\nt{hexadecimal-floating-point-literal}缩
放值为$s \times 2^e$。

\begin{example}
  \begin{sloppypar}
    \nt{floating-point-literal} \tm{49.625}和\tm{0xC.68p+2}具有相同值。
    \nt{floating-point-literal} \tm{1.602'176'565e-19}和\tm{1.602176565e-19}具有
    相同值。
  \end{sloppypar}
\end{example}

\paragraph{} % 3
如果缩放值不在其类型可表示值范围内,则程序为\illform{}。否则,如果可表示,则
\nt{floating-point-literal}的值为缩放值,否则为以实现方式选择的最接近缩放值的更
大或更小可表示值。
