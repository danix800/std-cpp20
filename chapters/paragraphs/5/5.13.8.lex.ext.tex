% 5.13.8 User-defined literals [lex.ext]
\synsym{user-defined-literal}
  \synprd{\nt{user-defined-integer-literal}}
  \synprd{\nt{user-defined-floating-point-literal}}
  \synprd{\nt{user-defined-string-literal}}
  \synprd{\nt{user-defined-character-literal}}
\synsym{user-defined-integer-literal}
  \synprd{\nt{decimal-literal ud-suffix}}
  \synprd{\nt{octal-literal ud-suffix}}
  \synprd{\nt{hexadecimal-literal ud-suffix}}
  \synprd{\nt{binary-literal ud-suffix}}
\synsym{user-defined-floating-point-literal}
  \synprd{\nt{fractional-constant exponent-part\tsub{opt} ud-suffix}}
  \synprd{\nt{digit-sequence exponent-part ud-suffix}}
  \synprd{\nt{hexadecimal-prefix hexadecimal-fractional-constant
    binary-exponent-part ud-suffix}}
  \synprd{\nt{hexadecimal-prefix hexadecimal-digit-sequence
    binary-exponent-part ud-suffix}}
\synsym{user-defined-string-literal}
  \synprd{\nt{string-literal ud-suffix}}
\synsym{user-defined-character-literal}
  \synprd{\nt{character-literal ud-suffix}}
\synsym{ud-suffix}
  \synprd[]{\nt{identifier}}

\paragraph{} % 1
如果一个符号可以匹配\nt{user-defined-literal}和其他种类的\nt{literal},则该符号
当作后者。

\begin{example}
  \tm{123\_km}为\nt{user-defined-literal},而\tm{12LL}为\nt{integer-literal}。
\end{example}

\nt{user-defined-literal}中的\nt{ud-prefix}之前的语法非结束符取可匹配该结束符的
最长字符序列。

\paragraph{} % 2
一个\nt{user-defined-literal}当作一个字面值运算符或字面值运算符模板
(\ref{over.literal})的调用。为确定对给定的带有\nt{ud-suffix X}的
\nt{user-defined-literal}的该调用的形式,字面值后缀标识符为\nt{X}的
\nt{literal-operator-id}在\nt{L}的上下文中使用未限定名查询规则
(\ref{basic.lookup.unqual})进行查询。设\nt{S}为该查询找到的声明集合。\nt{S}不
应该为空集。

\paragraph{} % 3
如果\nt{L}是一个\nt{user-defined-integer-literal},设\nt{n}为不带\nt{ud-suffix}
的字面值。如果\nt{S}包含一个形参类型为\tm{unsigned long long}的字面值运算符,则
字面值\nt{L}当作形如
\begin{codeenv}
  \code{operator "" \nt{X}(\nt{n}ULL)}
\end{codeenv}
的调用。否则,\nt{S}应该包含一个原始字面值运算符或者一个数值字面值运算符模板
(\ref{over.literal}),但不能两者都有。如果\nt{S}包含一个原始字面值运算符,则字
面值\nt{L}当作形如
\begin{codeenv}
  \code{operator "" \nt{X}("\nt{n}")}
\end{codeenv}
的调用。否则(\nt{S}包含一个数值字面值运算符模板),\nt{L}当作形如
\begin{codeenv}
  \code{operator "" \nt{X}<'\nt{c\tsub{1}}', '\nt{c\tsub{2}}', ...
        '\nt{c\tsub{k}}'>()}
\end{codeenv}
的调用,其中\nt{n}为源码字符序列$c_1c_2\ldots c_k$。

\begin{note} % 1
  序列$c_1c_2\ldots c_k$只能包含基本源码字符集中的字符。
\end{note}

\paragraph{} % 4
如果\nt{L}是一个\nt{user-defined-floating-literal},设\nt{f}为不带\nt{ud-suffix}
的字面值。如果\nt{S}包含一个形参类型为\tm{long double}的字面值运算符,则字面值
\nt{L}当作形如                                                                \\
\mbox{\qquad \tm{operator "" \nt{X}(\nt{f}L)}}                                \\
的调用。否则,\nt{S}应该包含一个原始字面值运算符或一个数值字面值运算符模板
(\ref{over.literal}),但不能两者都有。如果\nt{S}包含一个原始字面值运算符,则
\df{字面值} \nt{L}当作形如                                                    \\
\mbox{\qquad \tm{operator "" \nt{X}("\nt{f}")}}                               \\
的调用。否则(\nt{S}包含一个数值字面值运算符模板),\nt{L}当作形如            \\
\mbox{\qquad \tm{operator "" \nt{X}<'\nt{c\tsub{1}}', '\nt{c\tsub{2}}', ...
  '\nt{c\tsub{k}}'>()}}                                                       \\
的调用,其中\nt{f}为源码字符序列$c_1c_2 \ldots c_k$。

\begin{note} % 2
  序列$c_1c_2 \ldots c_k$只能包含基本源码字符集的字符。
\end{note}

\paragraph{} % 5
如果\nt{L}是一个\nt{user-defined-string-literal},设\nt{str}为不带\nt{ud-suffix}
的字面值,\nt{len}为\nt{str}中编码单元的个数(即不含结尾零字符的长度)。如果
\nt{S}包含一个带有非类型模板形参(\nt{str}是它的一个\wellform{}的
\nt{template-argument})的字面值运算符模板,则字面值\nt{L}当作形如            \\
\mbox{\qquad\tm{operator "" \nt{X}(\nt{str})}}                                \\
的调用。否则,字面值\nt{L}当作形如                                            \\
\mbox{\qquad\tm{operator "" \nt{X}(\nt{str}, \nt{len})}}                      \\
的调用

\paragraph{} % 6
如果\nt{L}是一个\nt{user-defined-character-literal},设\nt{ch}为不带
\nt{ud-suffix}的字面值。\nt{S}应该包含一个字面值运算符(\ref{over.literal}),其
唯一的形参具有\nt{ch}的类型,且\nt{L}当作形如                                 \\
\mbox{\qquad\tm{operator "" \nt{X}(\nt{ch})}}                                 \\
的调用。

\paragraph{} % 7
\begin{example} % 2
  \begin{lstlisting}
    long double operator "" _w(long double);
    std::string operator "" _w(const char16_t*, std::size_t);
    unsigned operator "" _w(const char*);
    int main() {
      1.2_w;      // calls operator "" _w(1.2L)
      u"one"_w;   // calls operator "" _w(u"one", 3)
      12_w;       // calls operator "" _w("12")
      "two"_w;    // error: no applicable literal operator
    }
  \end{lstlisting}
\end{example}

\paragraph{} % 8
在翻译阶段\ref{tpit6}(\ref{lex.phases}),相邻\nt{string-literal}被连接起来且
\nt{user-defined-string-literal}为此而当作\nt{string-literal}参与连接。连接时去
除\nt{ud-suffix},连接过程按\ref{lex.string}所述进行。在阶段\ref{tpit6}结束时,
如果一个\nt{string-literal}由涉及至少一个\nt{user-defined-string-literal}的连接
产生,则所有参与的\nt{user-defined-string-literal}应该具有相同的\nt{ud-suffix},
且该后缀用于连接结果。

\paragraph{} % 9
\begin{example} % 3
  \begin{lstlisting}
    int main() {
      L"A" "B" "C"_x;  // OK: same as L"ABC"_x
      "P"_x "Q" "R"_y; // error: two different ud-suffixes
    }
  \end{lstlisting}
\end{example}
