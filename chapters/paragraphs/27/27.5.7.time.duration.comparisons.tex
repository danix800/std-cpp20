% 27.5.7 Comparisons [time.duration.comparisons]
\paragraph{}
<++>

\paragraph{}
<++>

\paragraph{}
<++>

\paragraph{}
<++>

\paragraph{}
<++>

\paragraph{}
<++>

\paragraph{}
<++>

